\usemodule[tugboat]

\input tugboat.dates

% From Karl's email
% This issue is not being produced
% on letter paper, but in the "jacow" size that Volker Schaa has used.
% That is:
%   \pdfpagewidth=210 true mm
%   \pdfpageheight=11 true in
%   \hoffset = -2.95mm


\definepapersize[jacow][width=210mm, height=11in]
\setuppapersize[jacow][jacow]
\setuplayout[backspace=\dimexpr 6pc-2.95mm \relax,
             cutspace=\dimexpr 6pc-2.95mm \relax]

% \showlayout

\def\tinyskip{\vskip 2pt plus2pt}
\edef\tubpage{\the\count0 }
\hfuzz=.7pt

\doifmodeelse{aditya}
 {\let\myparindent\relax % This interferes on my machine
  \let\mystrut\relax}%This is for testing
 {\def\myparindent{\leavevmode\kern20pt }
  \def\mystrut{\vrule width0pt height2.6ex depth0pt }}

\setvariables
  [tugboat]
  [type=proceedings,
   columns=yes,
   grid=no]

\setupcolumns[balance=no]

\setvariables
   [tugboat]
   [year=\tubyear,
    volume=\tubvol,
    number=\tubnum,
    page=\tubpage]

\setvariables
  [tugboat]
  [   title={\CONTEXT\ basics for users: Table macros II},
   subtitle={Table macros continued}, % Subtitle does not work :-(
     author={Aditya Mahajan},
    address={University of Michigan},
      email={adityam (at) umich dot edu},
  ]

% Macros {{{

\def\gettablebuffer
  {\dosingleargument\dogettablebuffer}

\def\dogettablebuffer[#1]%
  {\startlinecorrection
    \midaligned
      \bgroup
      \doifelsenothing{#1}
        {\getbuffer}%
        {\getbuffer[#1]}%
      \egroup
    \stoplinecorrection}

\def\typetablebuffer
  {\dosingleargument\dotypetablebuffer}

\def\dotypetablebuffer[#1]%
  {\blank[small]%
   \startpacked
    \typebuffer[#1]%
    \stoppacked
   \blank[small]}
    
\hyphenation
 {start-table
  start-tables
  start-tabulate
  start-line-table
  stop-table
  stop-tables
  stop-tabulate
  stop-line-table
  Wich-ura
  mne-mon-ic}

% }}}

\starttext
\StartArticle

\StartAbstract
  \hbox to \hsize{\hfil This article explains some of the basic features of \mono{table} macros in
  \CONTEXT.\hfil}
\StopAbstract

\section Introduction %{{{1

In the last article, I presented some basic features of the \mono{table}
macros in \CONTEXT. In this article I will present additional features of
these macros. These two articles cover the most frequently used features.
There are other hooks for more advanced tweaking; some are covered in the
\CONTEXT\ beginner's manual~\cite[beginnersmanual]; others require reading
\filename{core-tab.tex}~\cite[core-tab].  A future article in this series may
touch upon those features.


\section Specifying font style and color of columns %{{{1

\startbuffer[pre-1]
\starttable[|lf{\bf}|c|]
\stopbuffer

\startbuffer[body]
  \NC Year \NC Production (in 1,000 units) \NC \AR
  \HL
  \NC 1990 \NC 20 \NC \AR
  \NC 1991 \NC 50 \NC \AR
\stoptable
\stopbuffer

Sometimes you want the entire column to be set in a particular font. For
example, suppose we want to produce the following table:
\gettablebuffer[pre-1,body]

One way to do this is to mark the first column in each row by \type{\bf}. This
is time consuming and difficult to change. \CONTEXT\ tables support an \mono{f}
key that can be used in the table preamble to set the font for the entire
column.  The preamble of the previous example was 
\typetablebuffer[pre-1]

\noindent 
Here \type{f{\bf}} tells \CONTEXT\ to typeset the first column of the table
with font style \type{\bf}. You can use any font style command with the
\mono{f} key. Some of the more frequently used font commands have been given a
key of their own. These are:

\midaligned{
\starttable[|lT|l|l|] 
  \NC B \NC Bold    \NC equivalent to \mono{f\char`{\letterbackslash bf\char`}} \NC \AR
  \NC I \NC Italic  \NC equivalent to \mono{f\char`{\letterbackslash it\char`}} \NC \AR
  \NC S \NC Slanted \NC equivalent to \mono{f\char`{\letterbackslash sl\char`}} \NC \AR
  \NC R \NC Roman   \NC equivalent to \mono{f\char`{\letterbackslash rm\char`}} \NC \AR
  \NC T \NC Teletype\NC equivalent to \mono{f\char`{\letterbackslash tt\char`}} \NC \AR
\stoptable
}

So, we could also have written the preamble of the previous example as
\type{\starttable[|lB|c|]}.

\section Changing the formatting of a cell %{{{1

In some tables, the header (first row of the table) needs to be bold and
center aligned, while the rest of the rows are left aligned. For example
\startbuffer
\starttable[|l|l|]
  \NC \REF[cB]{Name} \NC \REF[cB]{Position} \NC \AR
  \HL
  \NC Someone        \NC An important person \NC \AR
  \NC Someone else   \NC A really important person \NC \AR
\stoptable
\stopbuffer
\gettablebuffer

\startbuffer
\starttable[|l|l|]
  \NC \REF[cB]{...} \NC \REF[cB]{...} \NC \AR
  \HL
  \NC ...  \NC ... \NC \AR
  \NC ...  \NC ... \NC \AR
\stoptable
\stopbuffer

\noindent
The input for this table is:
\typetablebuffer

\noindent
Notice that the table preamble says \type{|l|l|}, that is, both columns should
be left aligned. In the first row we say \type{\REF[cB]{...}}, which changes
the formatting of that cell to \mono{cB}, that is, center aligned and bold.
\type{\REF} is a short form of \type{\ReFormat}; both macros
can be used for changing the format of the current cell. The general syntax of
the command is \type{\REF[keys]{column content}} where \type{keys} can
be any of the valid formatting keys accepted in the table preamble.

To change only the alignment of the current cell, you can use
\type{\JustCenter}, \type{\JustLeft}, and \type{\JustRight}, which stand for
\quote{justify center}, \quote{justify left}, and \quote{justify right},
respectively.\footnote{Most macros in \CONTEXT\ use the word \quote{align}.
These macros come from the \TABLE\ package, which uses the word
\quote{justify}.} 


\section Columns containing math %{{{1

Suppose we want an entire column to to be in math mode. For example,
\startbuffer
\starttable[|cm|cM|l|]
  \NC \REF[c]{Constant} \NC \REF[c]{Series}
  \NC \REF[c]{Value} \NC \AR
  \HL
  \NC \pi \NC 3 \sum_{n=0}^{\infty} 
      \frac {(2n)!} {n!^2 (2n+1) 16^n}
  \NC 3.1415926\dots \NC \AR
  %
  \NC e   \NC \sum_{n=0}^{\infty} 
      \frac 1{n!}
  \NC 2.7182818\dots \NC \AR
\stoptable
\stopbuffer
\gettablebuffer

\noindent In this case, the first two columns are in math mode (The second is
actually in display math mode).  We can manually surround each entry by
\type{$}; however, \CONTEXT\ provides two \quote{math column} keys: \mono{m}
sets the column in inline math mode, and \mono{M} sets the column
in display math mode. The above example was thus keyed in as
\typetablebuffer

\noindent The first column is in inline math mode (\mono{m} key), and the
second column is in display math mode (\mono{M} key).  Notice that I have used
\type{\REF[c]{...}} in the first row, so the
headings are not in math mode. 

\section Numeric columns %{{{1

Tables containing statistical data need the data to be aligned at the decimal
place. \CONTEXT\ provides two keys for this: 
\mono{n} displays the column in text mode, while \mono{N} displays it in
math mode. Both keys take a space-delimited argument of the form \type{x.y}
where \type{x} is the number of digits before the decimal and \type{y} is the
number of digits after the decimal. For example, to get the following table
(adapted from Tobias Oetiker's ``The not so short introduction to \LATEX''):

\startbuffer
\starttable[|cm|n5.4 |]
  \NC \REF[c]{Pi expression} 
  \NC \REF[c]{Value} \NC \AR
  \HL
  \NC \pi               \NC 3.1416   \NC \AR
  \NC \pi^{\pi}         \NC 36.46    \NC \AR
  \NC (\pi^{\pi})^{\pi} \NC 80,662.7 \NC \AR
  \HL
\stoptable
\stopbuffer

\gettablebuffer

\noindent I keyed in

\typebuffer

\noindent The key \type{|n5.4 |} (notice the space at the end) means that we
want five digits before the decimal and four digits after the decimal.

Some European countries use a comma as a decimal separator. This can be done
using the \mono{q} and \mono{Q} keys. They take a space-delimited argument
of the form \type{x,y} which has the same meaning as the argument of \mono{n}
and \mono{N} keys. So, to get this table

\startbuffer
\starttable[|cm|q5,4 |]
  \NC \REF[c]{Pi expression} 
  \NC \REF[c]{Value} \NC \AR
  \HL
  \NC \pi               \NC 3,1416   \NC \AR
  \NC \pi^{\pi}         \NC 36,46    \NC \AR
  \NC (\pi^{\pi})^{\pi} \NC 80.662,7 \NC \AR
  \HL
\stoptable
\stopbuffer

\gettablebuffer

\noindent I keyed in

\typebuffer

An \mono{n} or \mono{N} column must contain a dot; a \mono{q} or \mono{Q}
column must contain a comma. For cells that do not contain a dot or comma
(for example, the headings of the table) we can use \type{\REF} to change the
formatting. The \TEX\ primitive \type{\omit} can be used to leave the cell
empty.

\section[sec:spanning] Spanning multiple columns and rows %{{{1

In table heads, one often needs a cell spanning multiple columns. \CONTEXT\
provides the \type{\use} macro to do this. This macro takes an
argument specifying the number of columns to span.  For example, to use five
columns, we can use \type{\use{5}}. The macros \type{\TWO}, \type{\THREE},
\type{\FOUR}, \type{\FIVE}, \type{\SIX} are shortcuts to span the corresponding
number of columns. 

By default, when spanning multiple columns, the formatting keys of the last
spanned column are in effect. We can use \type{\REF} to change the formatting.
\CONTEXT\ also provides \type{\Use} (note the uppercase \mono{U}) to span
multiple columns and also set the formatting: to span three columns and
make the content center aligned  we can use \type{\Use{3}[c]{content}}. 

The support for spanning multiple rows is more limited. There are two commands
\type{\Lower} and \type{\Raise} that can lower or raise the contents of the
cell. There are two forms of these commands: \type{\Raise{5}{content}} which
raises the content by 5 times 1/12th of the line height; and
\type{\Raise(dimen){content}} which raises the content by \type{dimen} (which
can be any valid \TEX\ dimension). 

The most common usage of spanning multiple rows is spanning two rows in table
heads. For that, we can use \type{\LOW}, which lowers the current cell by
half of the line height, making it visually centered between the two rows.

Here is a table showing both column and row spanning
(example adapted from Andy Roberts' \LATEX\ tutorial~\cite[roberts]):
\tinyskip

\startbuffer[teams]
\starttable[|l|l|l|]
  \HL
  \VL \Use{3}[c]{Team Sheet}      \VL \AR
  \HL
  \VL Goal Keeper  \VL GK 
              \VL Paul Robinson   \VL \AR
  \HL
  \VL \Lower(1.5\lineheight){Defenders} \VL 
     LB \VL Lucus Radebe    \VL \AR \VL \VL
     DC \VL Michael Duberry \VL \AR \VL \VL
     DC \VL Dominic Matteo  \VL \AR \VL \VL
     RB \VL Didier Domi     \VL \AR
  \HL
\stoptable
\stopbuffer
\gettablebuffer[teams]

\noindent This was typed as

\typetablebuffer[teams]

\noindent In the first row we use \type{\Use{3}[c]{...}} to span three columns and
make the cell center aligned. In the first column of the third row, we use
\type{\Lower(1.5\lineheight){...}} to lower the cell so
that it appears to be visually centered in the last four rows.

\section Controlling space between columns %{{{1

By default, there is a 0.5\,em (usually about half
the current font size) space between the columns.
We can change this using the \mono{o} and the \mono{s} keys. The \mono{o} key
changes the space on the right of the current column; the \mono{s} key changes
the space of all the following columns until the next \mono{o} or \mono{s} key.

There are two ways of specifying the parameters of the \mono{o} and \mono{s} keys.
The first is in integer multiples of 0.5\,em: \type{s{n}} makes the space equal
to \mono{n} times 0.5\,em. So, to get a space of 1.5\,em between columns we can
use \type{\starttable[s{3}|l|l|]}. The second way is to specify the distance as
an arbitrary \TEX\ dimension. So, we could also have used
\type{\starttable[s(1.5em)|l|l|]}. Notice that in the first case, the argument
is given in curly brackets;\footnote{Actually, this is one argument according to
\TEX's parsing rule. So, for single digit arguments, we can omit the curly
brackets.} in the second, the argument is given in parentheses.

\myparindent
It is also possible to add padding (kerning) to the left and/or right of each column.
The key \mono{i} adds padding to the left, \mono{j} adds padding to the right,
and \mono{k} adds padding to both the right and the left.\footnote{We are
really running out of letters of the alphabet!} The amount of padding is
specified in the same way as in the \mono{o} and the \mono{s} keys: either in
multiples of 0.5\,em, or as arbitrary dimensions.

\myparindent
A combination of these keys can be used to force the \mono{table} macros to
produce tables as recommended by the \filename{booktabs} package~\cite[booktabs]. For
example

\startbuffer
\setuptables[rulethickness=0.03em]
\starttable[s0|l|i2l|i2r|]
\HL[3]
\NC \Use2[c]{Item}         \NC           \NC \AR
\DL[2]                     \DC               \DR
\NC Animal \NC Description \NC Price (\$)\NC \AR
\HL[2]
\NC Gnat   \NC per gram    \NC 13.65     \NC \AR
\NC        \NC each        \NC  0.01     \NC \AR
\NC Gnu    \NC stuffed     \NC 92.50     \NC \AR
\NC Emu    \NC stuffed     \NC 33.33     \NC \AR
\HL[3]
\stoptable
\stopbuffer

\typetablebuffer

\noindent gives

{\gettablebuffer}

Notice that in this case, the horizontal lines do not extend beyond the table.
The half line\footnote{In these articles, I have only talked about \type{\HL}
and not explained how to get {\em division lines} between rows. This is
explained in the \CONTEXT\ manual~\cite[beginnersmanual].} after the first row
extends only until the end of the second column. Compare this with the table that we
get from \type{\starttable[|l|l|r|]}:

\startbuffer
\setuptables[rulethickness=0.03em]

\starttable[|l|l|r|]
\HL[3]
\NC \Use2[c]{Item}         \NC           \NC \AR
\DL[2]                     \DC               \DR
\NC Animal \NC Description \NC Price (\$)\NC \AR
\HL[2]
\NC Gnat   \NC per gram    \NC 13.65     \NC \AR
\NC        \NC each        \NC  0.01     \NC \AR
\NC Gnu    \NC stuffed     \NC 92.50     \NC \AR
\NC Emu    \NC stuffed     \NC 33.33     \NC \AR
\HL[3]
\stoptable
\stopbuffer

{\gettablebuffer}

We will not go into the details of coaxing and beating \mono{table} macros into
producing tables like the \filename{booktabs} package: they were never designed
for that task. We can achieve the simpler parts of the \filename{booktabs}
recommendation, but for more complicated things such as \type{\cmidrule}, the
\mono{table} macros do not have enough hooks.

% Aditya: Manual footnote placement
% This does not work
% \setupfootnotes[before={\setbox\scratchbox=\vbox\bgroup},
%                 after={\egroup\raise-5cm\box\scratchbox}] 
%                 % Cant use raise in vertical mode

\setupfootnotes[before={\vskip-3cm}] %The actual value does not seem to matter.

To achieve lines that get trimmed at the edge of the table, we can use
\type{\starttable[o0|l|l|ro0|]}, \footnote{Wait a minute! To which column does the
first \mono{o0} correspond? The table consists of a virtual column at the
left edge, typically for drawing a vertical line there. The key \mono{o0} in
the beginning of the preamble sets the width of this virtual column to be zero.}
which gives:\tinyskip

\startbuffer
\setuptables[rulethickness=0.03em]

\starttable[o0|l|l|ro0|]
\HL[3]
\NC \Use2[c]{Item}         \NC           \NC \AR
\DL[2]                     \DC               \DR
\NC Animal \NC Description \NC Price (\$)\NC \AR
\HL[2]
\NC Gnat   \NC per gram    \NC 13.65     \NC \AR
\NC        \NC each        \NC  0.01     \NC \AR
\NC Gnu    \NC stuffed     \NC 92.50     \NC \AR
\NC Emu    \NC stuffed     \NC 33.33     \NC \AR
\HL[3]
\stoptable
\stopbuffer

{\gettablebuffer}

\noindent The division line in this case extends to the middle of the second
column. If the table does not have division lines, adding \type{o0} in the
beginning and the end of the table preamble is usually sufficient.

\section Controlling space between rows %{{{1

The \mono{table} macros do not provide much control over space between rows of
the table. You can have loose or tight tables by changing the \mono{distance}
option of \type{\setuptables}. The \mono{distance} option takes four values:
\mono{none}, \mono{small}, \mono{medium}, and \mono{big}. The default is
\mono{medium}. For example, let's reconsider the table of \in
Section[sec:spanning]. With \type{\setuptables[distance=none]}, we get
\tinyskip

{\setuptables[distance=none]
 \gettablebuffer[teams]}

\noindent while with \type{\setuptables[distance=big]} we get
\tinyskip

{\setuptables[distance=big]
 \gettablebuffer[teams]}

\noindent Play around with these values to find out what value of
\mono{distance} you prefer. It is possible to get more control using the
\type{\OpenUp} macro of the \TABLE\ package, but there is no interface for
that.  See the discussion on the mailing list~\cite[mailinglist] for an
example.

% }}}

\section Remembering preambles %{{{1

Often one has several tables which need to have similar formatting.
Repeating the table preamble in each case is error-prone. \CONTEXT\ provides
\type{\definetabletemplate} which can be used to specify a table preamble which
can be reused later. For example, we can say
\starttyping
\definetabletemplate[booktabs][o0|l|l|ro0]
\stoptyping

\noindent Then we can invoke this preamble by 
\starttyping
\starttable[booktabs]
\stoptyping

\section Other features %{{{1

When I started writing these articles on the \mono{table} macros, I thought that one
article would be enough. About halfway through the first article I realized that I would 
need more than one article. Now I find that even two are not enough. There are lots
of things that I have not even touched; using color in tables and breaking the table
across pages are the most important omissions. These will have to wait for a later
article in this series. Next issue, we will look at something different.


\section References

{
\vskip-10pt
\rightskip = 0pt plus4em \hyphenpenalty=10000
\hfuzz=5pt

\startbibliography[inbetween={\blank[4pt]}]
  \item[booktabs] Simon Fear and Danie Els, \quotation{Publication quality tables in \LATEX}, 
    \mono{http://www.ctan.org/\break tex-archive/macros/latex/contrib/booktabs}
  \item[beginnersmanual] Hans Hagen, \CONTEXT: an excursion. \hfil\break
    \mono{http://www.pragma-ade.com/show-man-1.htm}
  \item[core-tab] Hans Hagen, \CONTEXT\ core macros\Dash \TABLE\break embedding.
    \mono{http://www.logosrl.it/context/\break modules/current/%
          singles/\rlap{core-tab\_ebook.pdf}}
  \item[roberts] Andy Roberts, \LaTeX\ Tutorials\Dash Tables. \hfil\break
    \mono{http://www.andy-roberts.net/misc/latex/\hfil\break latextutorial4.html}
  \item[mailinglist] Discussion on the \CONTEXT\ mailing list,
        \mono{http://archive.%
          contextgarden.net\break/message/20070806.011325.%
          5a938ae7.en.html}
\stopbibliography
}

\vfill


% }}}
\StopArticle
\stoptext
