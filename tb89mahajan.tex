\usemodule[tug-02]
\input tugboat.dates
\setfirstpageafter{hellstrom}

\def\myparindent{20pt}
\def\myindent{\leavevmode\hbox to \myparindent{\hfil}}

\def\mystrut{\vrule width0pt height2.6ex depth0pt }

\setuplayout[lines=0] 

\definebodyfontenvironment    [8pt] [typing=8pt] %This is what footnotes use
\definebodyfontenvironment    [9pt] [typing=9pt] %8pt looks too small to me
\definebodyfontenvironment   [10pt] [typing=9pt]
\definebodyfontenvironment   [12pt] [typing=9pt]
\definebodyfontenvironment [14.4pt] [typing=9pt]

\setuptyping[style={\switchtobodyfont[typing]\tt}]


\setvariables
  [tugboat]
  [type=article,
   columns=yes,
   grid=no]

\setvariables
  [tugboat]
  [year=\tubyear,
   volume=\tubvol,
   number=\tubnum,
   page=\startpage]

\setvariables
  [tugboat]
  [   title={\CONTEXT\ basics for users: Font~styles},
   subtitle={Font Styles}, % Subtitle does not work :-(
     author={Aditya Mahajan},
    address={University of Michigan},
      email={adityam (at) umich dot edu},
  ]

\useURL[manuals][http://pragma-ade.com/show-man-1.htm]
\useURL[sources][http://www.logosrl.it/context/modules/]
\useURL[mailing][http://wiki.contextgarden.net/ConTeXt_Mailing_Lists]
\useURL[existingfonts][http://pragma-ade.com/general/manuals/showfont.pdf]
\useURL[newfonts][http://pragma-ade.com/general/manuals/mfonts.pdf]
\useURL[tfmetrics][http://pragma-ade.com/general/technotes/tfmetrics.pdf]

\starttext
\StartArticle

\StartAbstract
  This article presents a summary of different ways of changing font styles
   in \CONTEXT.
\StopAbstract

\vskip-6pt
\section Introduction

The \TUGBOAT\ editors recently invited me to write a regular column
in \TUGBOAT\  explaining some of the basic features of \CONTEXT. This column
is meant for \CONTEXT\ beginners, and will explain how basic elements of
\CONTEXT\ work. I will explain it from the practicable point of view, that is,
do this, and you will get this; to understand what is happening behind the
scenes you need to read the \CONTEXT\ manuals \footnote[manual]{\mystrut
{\tt \from[manuals]}}
and the \CONTEXT\ sources.\footnote{\mystrut {\tt \from[sources]}}

\myindent
In this first installment, I will discuss how to use the various font styles in
\CONTEXT. Fonts are one of the most complicated parts of \TEX. Fortunately,
the macro developers take care of the dirty stuff, and most of the user
interface is clean. Nevertheless, understanding the various options of the
user interface can be intimidating.  As a beginner, one does not want to know
all the nitty|-|gritty details, but just the basic features. We hope to
present these in this article.

In \CONTEXT\ there are five ways to switch fonts:
\blank[halfline]

\startitemize[n,joinedup,nowhite,after,intro]
  \item font style (\type{\rm}, \type{\ss}, etc.),
  \item font size  (\type{\tfa}, \type{\tfb}, etc.),
  \item alternative font style (\type{\bold}, \type{\sans}, etc.),
% \item body font switch (\type{\tenpoint}, \type{\eightpoint}, etc.)
  \item a complete font change (\type{\setupbodyfont},
    \type{\switchtobodyfont}).
\stopitemize
\vskip -6pt

I will briefly explain each of these.
%The rest of the article is
%going to assume that the font typefaces are set up correctly. 
%You do not need
%to worry about font typefaces unless you are adding support for a new font.
%Idris's article in this issue explains how to do that.

\section Font styles

There are three types of font families: serif, sans serif, and teletype.  To
switch between these families, use \type{\rm} for serif, \type{\ss} for sans
serif, and \type{\tt} for teletype.

Each of these families come in different
styles: upright, bold, italic, slanted, bold|-|italic, bold|-|slanted, and
small caps. To switch to a different style, use \type{\tf} for upright,
\type{\bf} for bold, \type{\it} for italic, \type{\sl} for slanted, \type{\bi}
for bold|-|italic, \type{\bs} for bold|-|slanted, and \type{\sc} for
small|-|capped.

You can generally combine font families and font styles,
so if you want to switch to bold sans serif, you can use either \type{\bf\ss} or
\type{\ss\bf}. 

\starthiding
These options are summarized in \in Table[tab:font-switches].
\startcomment
  The table looks weird as most fonts do not come out correct. I have managed
  to get bold teletype to work earlier, but I have no clue about bold italic
  sans. Is it missing from latin modern, or is it missing from context
  definitions. Since this is an introductory article, these details will be an
  overkill, so I am deleting the table :-(
\stopcomment

\placetable
    [tab:font-switches]
    [here,top,bottom]
    {Font switches for changing font styles}
    \bgroup%{{{
    \starttable[|ls(1em)|l|l|l|]
      \HL[3]
      \BL[4] \SR
      \NC \ReFormat[cB] {switch} 
      \NC \ReFormat[cB] {\tex{rm} roman}
      \NC \ReFormat[cB] {\tex{ss} sans serif}
      \NC \ReFormat[cB] {\tex{tt} typewriter}
      \NC \SR
      \HL[2]
      \NC \tex{tf} 
      \NC {\tf normal/upright}
      \NC {\tf\ss normal/upright}
      \NC {\tf\tt normal/upright}
      \NC \FR
      \NL
      \NC \tex{it}
      \NC {\it italics}
      \NC {\it\ss italics}
      \NC {\it\tt italics}
      \NC \MR
      \NL
      \NC \tex{sl}
      \NC {\sl slanted}
      \NC {\sl\ss slanted}
      \NC {\sl\tt slanted}
      \NC \MR
      \NL
      \NC \tex{bf}
      \NC {\bf bold}
      \NC {\bf\ss bold}
      \NC {\bf\tt bold}
      \NC \MR
      \NL
      \NC \tex{bi}
      \NC {\bi bold italic}
      \NC {\bi\ss bold italic}
      \NC {\bi\tt bold italic}
      \NC \MR
      \NL
      \NC \tex{bs}
      \NC {\bs bold slanted}
      \NC {\bs\ss bold slanted}
      \NC {\bs\tt bold slanted}
      \NC \LR
      \HL[3]
    \stoptable
    \egroup %}}}
\stophiding

There is a font switch \type{\em} to {\em emphasize} text. This is somewhat
special: it does automatic italic correction and changes the style depending
on the current font style. For example, if the current font style is upright,
\type{\em} switches to slanted; and if the current font style is slanted,
\type{\em} switches to upright.

\ConTeXt\ uses the Latin Modern fonts by default; these fonts look
similar to the original Computer Modern fonts, but have a much larger
character repertoire.  As it happens, in the Latin Modern (and Computer
Modern) fonts, the \slanted{slanted} font does not stand out from the
upright font enough for some tastes; so, many people prefer to use the
\italic{italic} font for emphasis. To do that use

\vskip-6pt
\starttyping
\definebodyfontenvironment[default][em=italic]
\stoptyping
\vskip-6pt

A font switch remains valid for the rest of the group. So, if you want to
temporarily switch to a different font, use the font style command inside a
group. The easiest way to start a group is to enclose the text within braces
(also called curly brackets), for example

\startbuffer
This is serif text
{\ss This is sans serif}
{\tt and this is typewriter}
\stopbuffer

\vskip-6pt
\typebuffer
\vskip-6pt

which gives (notice the braces in the above lines)

\vskip-6pt
\startlines
\getbuffer
\stoplines
\vskip-12pt

\section Font sizes

Occasionally one needs to change the font size. \CONTEXT\ provides two series
of commands for that. To increase the font you can use \type{\tfa} to scale
the font size by a factor of $1.2$, \type{\tfb} to scale by a factor of
$(1.2)^2 = 1.44$, \type{\tfc} to scale by $(1.2)^3 = 1.728$ and \type{\tfd} to
scale by $(1.2)^4 = 2.074$.

To decrease the font size, you can use \type{\tfx}
to scale the font by a factor of $0.8$ and \type{\tfxx} to scale by a factor
of $0.6$. The scale factors can be a function of the current font size and can
be changed by \type{\definebodyfontenvironment}.

For example, if you want
\type{\tfa} to be equal to 12pt when you are using 10pt font, and be equal to
14pt when you are using 11pt font, then add

\vskip-4pt
\starttyping
\definebodyfontenvironment [10pt] [a=12pt]
\definebodyfontenvironment [11pt] [a=14pt]
\stoptyping
\vskip-4pt

\noindent
The \type{\definebodyfontenvironment} command is described in detail in 
the \CONTEXT\ manual and the \filename{font-ini.tex} source file.

Font size can be combined with font styles. As a shortcut, you can use
\type{\bfa} to get bold font scaled by $1.2$, \type{\bfx} to get a bold font
scaled by $0.8$ and similar commands for other font styles.

These font size switches are meant for changing the font size of a few
words: they do not change the interline spacing and math font sizes. So, if
you want to change the font size of an entire paragraph, use
\type{\switchtobodyfont} described below in \in Section[setupbodyfont]. However, it
is fine to use them as style directives in setup commands, that is, using them
as an option for \type{style=...} in any setup command that accepts the
\type{style} option.

\section Alternative font styles

While learning a document markup language like \CONTEXT, it can be hard to
remember all the commands. \CONTEXT\ provides easy to remember alternative
font styles. So for bold you can use \type{\bold}, for italic you can use
\type{\italic}, for slanted you can use \type{\slanted}, and so on. You can
probably guess what the following do:
\blank[halfline]

\starttable[|l|l|]
  \NC \type{\normal} 
  \NC \type{\slanted}
  \NC \NR
  \NC \type{\boldslanted}
  \NC \type{\slantedbold}
  \NC \NR
  \NC \type{\bolditalic}
  \NC \type{\italicbold}
  \NC \NR
  \NC \type{\small}
  \NC \type{\smallnormal}
  \NC \NR
  \NC \type{\smallbold}
  \NC \type{\smallslanted}
  \NC \NR
  \NC \type{\smallboldslanted}
  \NC \type{\smallslantedbold}
  \NC \NR
  \NC \type{\smallbolditalic}
  \NC \type{\smallitalicbold}
  \NC \NR
  \NC \type{\sans}
  \NC \type{\sansserif}
  \NC \NR
  \NC \type{\sansbold}
  \NC \type{\smallcaps} 
  \NC \NR
\stoptable
\blank[halfline]

\noindent
In addition, the commands \type{\smallbodyfont} and
\type{\bigbodyfont} can be used to change the font size.

These alternative font styles are pretty smart. You can either use them as
font style switches inside a group, or as a font changing command that takes an
argument. For example,
\startbuffer
This is {\bold bold} and so is \bold{this}.
\stopbuffer

\vskip-6pt
\typebuffer
\vskip-6pt

gives

\vskip-6pt
\startlines
\getbuffer
\stoplines

These alternative font styles can also be used for all \type{style=...}
options, and while using them as style options, you can just give the command
name, for example:

\vskip-6pt
\starttyping
\setuphead[section][style=bold] 
\stoptyping

\column
\section[setupbodyfont] Complete font change

If you need to change to a different font size and take care of interline
spacing, you can use \type{\switchtobodyfont}. For example, to switch to 12pt
you can use \type|\switchtobodyfont[12pt]|.

\CONTEXT\ provides two relative sizes, called `big' and `small'. So, to go to a
bigger font size, you can use \type|\switchtobodyfont[big]| and to go to a
smaller font size, \type|\switchtobodyfont[small]|. The exact
sizes that are used for big and small can be set using
\type|\definebodyfontenvironment|.

The \type{\setupbodyfont} command accepts all the same arguments as
\type{\switchtobodyfont}. The difference between the two is that
\type{\setupbodyfont} also changes the font for headers, footers and other
page markings, while \type{\switchtobodyfont} does not. So you should use
\type{\setupbodyfont} for global font definitions to apply to the whole document, and
\type{\switchtobodyfont} for local font changes. The effect of
\type{\switchtobodyfont} can be localized within a group as usual.


\vskip-3pt
\section Different typefaces

So far we have discussed style and size changes within a given typeface
family.  If you want to use a different typeface
altogether, such as Times or Palatino, the Pragma web site has recipes
covering all the commonly available typefaces,\footnote{\mystrut {\tt
\from[existingfonts]}} while a separate manual describes how to write
support for new typefaces.\footnote{\mystrut {\tt \from[newfonts]}} (For
the latter, see also Idris Hamid's article in this issue of
\slanted{TUGboat}.)

%\starttyping
%\definetypeface [palatino] [rm] [serif] [palatino] [default] [encoding=ec]
%\switchtotypeface [palatino] [10pt,rm]
%\stoptyping

\myindent
The recipes as given work with the standalone \ConTeXt\
distribution, but not with \TeX\ Live et al.%
\footnote{\mystrut {\tt \from[tfmetrics]} explains why \CONTEXT\ uses 
separate font metrics, and gives some differences between the 
sets.\vrule height0pt width0pt depth.6ex \hfuzz=9pt\baselineskip=10pt\par}
To use the recipes with other distributions, try adding one of
\type|\usetypescript[berry][ec]| or \type|\usetypescript[adobekb][ec]|.

\vskip-3pt
\section Conclusion

There are many other ways of choosing font styles in \CONTEXT.  If these
basic styles do not satisfy your needs, have a look at the manual
\note[1], or ask on the \CONTEXT\ mailing list.\footnote{\mystrut
\hfuzz=6pt {\tt \from[mailing]}\par}

\setlastpage{1}

\StopArticle
                                     
\stoptext
