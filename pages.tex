\usemodule[tugboat]

\def\HTML{\acro{HTML}}

\setvariables
  [tugboat]
  [type=article,
   columns=yes,
   grid=no]

% \setvariables
%    [tugboat]
%    [year=\tubyear,
%     volume=\tubvol,
%     number=\tubnum,
%     page=\tubpage]

\setvariables
  [tugboat]
  [   title={\CONTEXT\ basics for users: Setting paper sizes},
     author={Aditya Mahajan},
    address={},
      email={adityam (at) ieee dot org},
  ]


% \setuppapersize[letter][letter]

\starttext
\StartArticle 
\StartAbstract
\StopAbstract

\section {Introduction}

Separation of content and presentation is the biggest selling point of markup
based document preparation systems. \TEX\ advocates will tell you to concentrate
on your content, and let the style writers worry about layout. This is good
advice. But occasionally you need to worry about layout. And changing layout can
be intimidating. The most visually obvious parts of a layout is paper size,
margins, and fonts. If you get these three things right, you are almost half way
through with the layout. In this article, I will explain how to set up the paper
size and margins|<|the page layout|>| in \CONTEXT. 

\section {Setting the paper size}

Plain \TEX\ and \LATEX\ were primarily developed in the US. So, they default to
letter paper, which is the standard paper size in the US. \CONTEXT\ was developed
in the Netherlands. So, it defaults to A4 paper, which is the
standard paper size in Europe (and almost everywhere else in the world).
Changing paper size is easy, if you want letter paper, you can type
\starttyping
  \setuppapersize[letter][letter]
\stoptyping

\noindentation
If you want A4 paper, you can type 
\starttyping
  \setuppapersize[A4][A4]
\stoptyping

You may wonder why we need to repeat the paper size twice. In most cases, these
are the same. You only need to use different arguments if you want to print on a
bigger paper and trim it later. As this is a rather specialized task, I will
not go into the details. 
\footnote{
Briefly, the first argument sets 
the paper size of the typeset page. The second argument sets the paper size of
the printed page. For example, if you want to typeset a A5 paper and print it on
a A4 paper, you can type
\starttyping
  \setuppapersize[A5][A4]
\stoptyping
}
You can look at Chapter~3 of the \CONTEXT\ user manual~\cite[manual] for 
details. 

You can use \type{\definepapersize} to define specific paper size and print size
combinations for later reuse. For example, if you type 

\starttyping
  \definepapersize[A4paper]     [A4]     [A4]
  \definepapersize[letterpaper] [letter] [letter]
\stoptyping

\noindentation
Then you can use

\starttyping
  \setuppapersize[A4paper]
\stoptyping
\noindentation
to get A4 paper and
\starttyping
  \setuppapersize[letterpaper]
\stoptyping
\noindentation 
to get letter paper.

If you want to change the page orientation, say letter paper in landscape mode,
you can type
\starttyping
  \setuppapersize[letter,landscape]
                 [letter,landscape]
\stoptyping

\noindentation
Other page orientations are possible, including mirroring and rotating the page.
Since there are also a bit specialized, I will not go into the details in this
article, and refer the inquisitive reader to Chapter~3 of the \CONTEXT\ user
manual~\cite[manual].

\section {Predefined paper sizes}

Many paper sizes are predefined in \CONTEXT. These include:
\startitemize[nowhite]
  \item \type{letter}, \type{ledger}, \type{tabloid}, \type{legal},
    \type{folio}, and \type{executive} sizes from the North American paper
    standard;
  \item sizes \type{A0} -- \type{A10}, \type{B0} -- \type{B10}, and
    \type{C0} -- \type{C10} from the A, B, and C series of the ISO~216 standard; 
  \item sizes \type{RA0} -- \type{RA4} and \type{SRA0} -- \type{SRA4} from the RA
    and SRA series of ISO-217 paper standard;
  \item sizes \type{C6/C5}, \type{DL}, and \type{E4} from ISO-269 standard
    envelope sizes;
  \item \type{envelope 9} -- \type{envelope 14} sizes from the American
    postal standard;
  \item sizes \type{G5} and \type{E5} from the Swedish SIS~014711 standard.
    These are used for Swedish theses; 
  \item size \type{CD} for CD covers;
  \item size \type{S3} -- \type{S6}, \type{S8}, \type{SM}, and \type{SW} for
    screen sizes. These sizes are useful for presentations.
    \type{S3} -- \type{S6} and \type{S8} have an aspect ratio of $4\colon 3$.
    \type{S3} is 300pt wide, \type{S4} is 400pt wide, and so on. \type{S6} is
    almost as wide as a \type{A4} paper. \type{SM} and \type{SW} are for medium
    and wide screens; they have the same height as \type{S6};
  \item a few more paper sizes, which I will not mention here. See
    \filename{page-lay.tex} for details.
\stopitemize

\section {Defining your own paper size}

The predefined paper sizes in \CONTEXT\ cannot fit all needs. In that case, it
is possible to define your own paper size. Suppose you want to print on a paper
that is 50mm wide and 100mm wide. A new paper size needs to have a name, and,
for the lack of a better term, I will call this paper size \type{exotic}. This
paper size can be defined by typing
\starttyping
  \definepapersize[exotic]
                  [width=50mm, height=100mm]
\stoptyping
\noindentation
Then, it can be used as any other predefined paper size
\starttyping
  \setuppapersize[exotic][exotic]
\stoptyping

Notice that this macro, \type{\definepapersize}, is the same macro that was also
used for defining paper size and print size combinations. When you use this
macro, \CONTEXT\ tests for number of arguments and for key value pairs in the
second argument to determine which of the two versions was meant.

\section {Changing paper size within a document}

To get one page (containing a table or a figure) in landscape
mode, you can use \type{\adaptpapersize}. For example,

\starttyping
\definepapersize[main] [A4] [A4]
\definepapersize[extra][A4,landscape]
                       [A4,landscape]

\setuppapersize[main]
\starttext
    Page 1. Potrait \page
    Page 2. Potrait \page
    \adaptpapersize[extra]
    Page 3. Landscape \page
    Page 4. Potrait \page
\stoptext
\stoptyping

If you have a full page figure that you want to include in a landscape paper,
you can combine \type{\adaptpapersize} with \CONTEXT's postponing mechanism.

\starttyping
  \startpostponing
    \adaptpapersize[extra]
    \placefigure
      [here]
      [fig:reference]
      {The caption of the figure}
      {\extrenalgraphics[full-page-figure]}
    \page
  \stoppostponing
\stoptyping

The \type{postponing} environment postpones the figure until the next
page. When the new page starts, \type{\adaptpapersize} changes to landscape
mode. The \type{\page} is the end is to make sure even if some place is
remaining on the page, we do not get any text there.

To get more than one page in landscape, you can reuse \type{\setuppapersize}.
For example,

\starttyping
\definepapersize[main] [A4] [A4]
\definepapersize[extra][A4,landscape]
                       [A4,landscape]

\setuppapersize[main]
\starttext
    Page 1. Potrait \page
    Page 2. Potrait \page
\setuppapersize[extra]
    Page 3. Landscape \page
    Page 4. Landscape \page
\setuppapersize[main]
    Page 5. Potrait \page
    Page 6. Potrait \page
\stoptext
\stoptyping

\section {Layout --- the basics}

In \CONTEXT, page layout is set using \type{\setuplayout}. This macro can be can
be intimidating to a new user because it has a lot of options. In this section I
will explain the options that are commonly used.

Page layout can be divided into two parts, horizontal and vertical. The
horizontal layout has three main options. These are, from left to right,
\type{backspace}, \type{width}, \type{cutspace}. The options \type{backspace}
and \type{cutspace} set the left and right margins,\footnote{By margin, I mean
the space between the edge of the paper and the edge of the text area.
\type{\setuplayout} has options called \type{leftmargin}, \type{rightmargin},
and \type{margin} which set a slightly different part of the layout.} while
\type{width} sets the main text width. All three of these options can take a
value of any valid \TEX\ dimension. In addition, the width can also take the
value of \type{fit} and \type{middle}. Lets look at some examples to understand
these options.

Suppose we want a layout with left and right margins of 1in. This can be
  set using
\starttyping
  \setuplayout
      [backspace=1in,
           width=middle]
\stoptyping
\noindentation
The option \type{width=middle} tells \CONTEXT\ to use same value for
\type{backspace} and \type{cutspace}. We could have also used
\starttyping
  \setuplayout
      [backspace=1in,
           width=fit,
        cutspace=1in]
\stoptyping
\noindentation
The option \type{width=fit} tells \CONTEXT\ to figure out the value of the
\type{width} from the paper width and the values of \type{backspace} and
\type{cutspace}. This form is useful when the left and right margins have
different values.

Like the horizontal layout, the vertical layout also has three main options.
There are, from top to bottom, \type{topspace}, \type{height},
\type{bottomspace}. The options \type{topspace} and \type{bottomspace} set the
top and bottom margins,\footnote{Again, by margin, I mean the space between the
edge of the paper and the edge of the text area.} while \type{height} sets the
main text height. Like their horizontal counterparts, all three of these options
can take a value of any valid \TEX\ dimension. In addition, the height can also
take the value of \type{fit} and \type{middle}. Lets look at some examples to
understand these options.

Suppose we want a layout with top and bottom margins of 1in. This can be
  set using
\starttyping
  \setuplayout
      [topspace=1in,
         height=middle]
\stoptyping
\noindentation
The option \type{height=middle} tells \CONTEXT\ to use same value for
\type{topspace} and \type{bottomspace}. We could have also used
\starttyping
  \setuplayout
      [topspace=1in,
         heigth=fit,
    bottomspace=1in]
\stoptyping
\noindentation
The option \type{height=fit} tells \CONTEXT\ to figure out the value of the
\type{height} from the paper height and the values of \type{topspace} and
\type{bottomspace}. This form is useful when the top and bottom margins has
different values.

Thus, to get a 1in margin on all sides, we can type
\starttyping
  \setuplayout
      [topspace=1in,
      backspace=1in,
          width=1in,
         height=in]
  \stoplayout
\stoptyping

\section {Double sided layout}

Books have a double sided layout, which means that odd and even pages have
mirrored layout (reflected about the center of the {\em spread}). A
double sided layout can be set up using
\starttyping
  \setuppagenumbering[alternative=doublesided]
\stoptyping
\noindentation
If you set up a double sided layout and, for some reason, want to revert to a
single sided layout, you can type
\starttyping
  \setuppagenumbering[alternative=singlesided]
\stoptyping

\section {Layout --- the details}



\section {References}

{
\vskip-10pt
\rightskip = 0pt plus4em \hyphenpenalty=10000
\hfuzz=5pt

\startbibliography[inbetween={\blank[4pt]}]
  \item[manual] Hans Hagen, \quotation{\em \CONTEXT\ the manual},\crlf
    \mono{http://www.pragma-ade.com/general/\crlf manuals/cont-eni.pdf}

\stopbibliography

}

\StopArticle
\stoptext
