% engine=luatex
\usemodule[abr-03]
\usemodule[tugboat]

\input tugboat.dates
\edef\tubpage{\the\count0}

\setupcolors[state=start]

\setvariables
  [tugboat]
  [type=article,
   columns=yes,
   grid=no]

 \setvariables
    [tugboat]
    [year=\tubyear,
     volume=\tubvol,
     number=\tubnum,
     page=\tubpage]

\setvariables
  [tugboat]
  [   title={Integrating Unicode and OpenType math\\in~\CONTEXT},
     author={Aditya Mahajan},
    address={},
      email={adityam (at) umich dot edu},
  ]

% I need a monospaced font that includes greek letters. The only font that I
% know that does this is dejavu sans mono. But, it looks horrible with LM. So, I
% am using cambria as the main font. If you know better alternatives, feel free
% to change the fonts.

%% \starttypescript [mono] [dejavu]
%%     %\setups[font:fallback:mono]
%%     \definefontsynonym [Mono]           [name:DejaVu Sans Mono]   
%%                        [features=none]
%%     \definefontsynonym [MonoItalic]     [name:DejaVu Sans Mono]    
%%                        [features=none]
%%     \definefontsynonym [MonoBold]       [name:DejaVu Sans Mono]      
%%                        [features=none]
%%     \definefontsynonym [MonoBoldItalic] [name:DejaVu Sans Mono]
%%                        [features=none]
%% \stoptypescript
%% 
%% \definetypeface [mainface] [rm] [serif] [cambria] [default]
%% %definetypeface [mainface] [rm] [serif] [charter] [default]
%% \definetypeface [mainface] [ss] [sans]  [cambria] [default]
%% \definetypeface [mainface] [tt] [mono]  [dejavu]  [default] [rscale=0.85]
%% \definetypeface [mainface] [mm] [math]  [cambria] [default, text=rm]
%% 
%% \setupbodyfont[mainface,10pt]
 
\starttypescript [mono] [cmunicode]
     %\setups[font:fallback:mono]
     \definefontsynonym [Mono]           [file:cmuntt]
                        [features=none,liga=off]
     \definefontsynonym [MonoItalic]     [file:cmuntt]
                        [features=none]
     \definefontsynonym [MonoBold]       [file:cmuntt]
                        [features=none]
     \definefontsynonym [MonoBoldItalic] [file:cmuntt]
                        [features=none]
\stoptypescript

\definetypeface[mainface][rm][serif][modern]   [default]
\definetypeface[mainface][ss][sans] [modern]   [default]
\definetypeface[mainface][tt][mono][cmunicode] [default]
\definetypeface[mainface][mm][math][modern] [default]

\setupbodyfont[mainface,10pt]

\StartAbstract
  In this article, I briefly explain the approach taken by \CONTEXT\ to
  integrate Unicode math with OpenType fonts. 
\StopAbstract

\starttext
\StartArticle

\section{A bit of history}

Since around 2000, \CONTEXT\ has supported Unicode math input. Under the
\type{utf-8} input regime (obtained with \type{\enableregime[utf-8]}), you could
type \type{$αβγ$} to get the Greek letters $\alpha\beta\gamma$. %$αβγ$. 
% Unicode math does not work with xetex!! Why??
This was achieved by lot of
macro jugglery; similar to the kind of macro jugglery needed for accents to
work without
having to type in the arcane plain \TEX\ accent macros. Basically, the input
regime ensured that inside math mode \type{α} got mapped to \type{\alpha}, and
the rest was taken care of by the usual font mappings that mapped \type{\alpha} to
the correct glyph in the correct font. However, these mappings were not clean. If
you defined a macro, say \type{\MACRO}, that took one argument, then
\type{\MACRO α} would not work; you had to type \type{\MACRO{α}}. This small detail of
grouping \UTF-8 input was a constant reminder that things were not so clean
underneath.

\LUATEX\ was designed to handle input encoding cleanly.
The engine only understands \UTF-8 encoding and provides enough hooks to 
implement other encodings. \CONTEXT\ \MKIV\ assumes that the input is 
either \UTF-8 or \UTF-16. The input can then be directly mapped to 
the correct glyph locations in TrueType and OpenType fonts.
However, handling of math input was much trickier, mainly because of the
effort needed to support OpenType math. So, for some time,
the handling of math fonts did not change: in math mode \type{α} was mapped to
\type{\alpha} and traditional \TEX\ font mapping ensured that \type{\alpha} was
mapped to the correct glyph in the correct font.

Around the beginning of this year, \CONTEXT\ \MKIV\ completely moved to Unicode
math. Thus, \type{0x1D6FC} (math italic small alpha) was mapped to the
same position in an OpenType math font.  If you type this Unicode character in
math mode, the output is $\alpha$. %With xetex, I have to cheat $𝛼$. 
For convenience, typing \type{0x03B1} (Greek
small letter alpha) in math mode gets mapped to \type{0x1D6FC}; as does the
macro \type{\alpha}. All this is transparent to the user, except when he
accidentally types \type{$\MACRO α$} and is pleasantly surprised not to get a
nasty error message. 

There are two steps involved in the integration of Unicode input with OpenType
math fonts: (i)~map the input characters or macros to the
correct Unicode character; (ii)~map the Unicode character to the correct
OpenType glyph at the correct size and with the correct kerning. The bulk of these
mappings are done by five files: \filename{char-def.lua},
\filename{math-ini.lua}, \filename{math-map.lua}, \filename{math-noa.lua} and \filename{math-vfu.lua}.
Below I explain briefly what these files do. Please take everything in this
article with a pinch of salt. I do not understand how OpenType math fonts work,
so specific details may be wrong. The main idea of this article is to convey the
gist of how things are done in \CONTEXT, and where to search if you want to know
specific details.


\section{char-def.lua}

Mapping \UTF-8 input characters to Unicode characters is straightforward\Dash{}the byte
sequence of the input character is the same as the Unicode slot. Thus, input
byte sequence \type{0x1D6FC} is the Unicode character \type{0x1D6FC}. However,
mapping macros to Unicode byte sequences is different. We need to explicitly
tell \CONTEXT\ that \type{\alpha} corresponds to the Unicode character
\type{0x1D6FC}. Furthermore, just getting the correct glyph is not sufficient
for typesetting mathematics. We also need to know the math class of the glyph,
so that \TEX\ can correctly position the characters.

All this information is stored as a
\LUA\ table in \filename{char-def.lua}. This is a huge file, initially
generated from Unicode tables and later updated manually from data present in
\CONTEXT\ files and elsewhere (it is still not complete). For example,
the entry for \type{0x1D6FC} is:
\starttyping
[0x1D6FC]={
 category="ll",
 description="MATHEMATICAL ITALIC SMALL ALPHA",
 direction="l",
 linebreak="al",
 mathclass="default",
 mathname="alpha",
 specials={ "font", 0x03B1 },
 unicodeslot=0x1D6FC,
},
\stoptyping
Amongst other things, this tells \CONTEXT\ that the macro \type{\alpha}
(indicated by \type{mathname="alpha"}) corresponds to this Unicode slot. It also
tells that the math class of this character is \type{default}. Similar details
are provided for a large fraction of the Unicode characters.

Some Unicode
characters correspond to more than one macro (with different math classes). One
such example is \type{0x007C}, which corresponds to \type{\arrowvert},
\type{\vert}, \type{\lvert}, \type{\rvert}, and \type{\mid}, each belonging to a
different math class. This Unicode character is represented as:
\starttyping
{
 adobename="bar",
 category="sm",
 cjkwd="na",
 contextname="textbar",
 description="VERTICAL LINE",
 direction="on",
 linebreak="ba",
 mathspec={
  { class="nothing",   name="arrowvert" },
  { class="delimiter", name="vert"      },
  { class="open",      name="lvert"     },
  { class="close",     name="rvert"     },
  { class="relation",  name="mid"       },
 },
 unicodeslot=0x007C,
},
\stoptyping
The different macros and their corresponding math classes are encoded as part of
a \type{mathspec} key. When the character \type{|} (\type{0x007C}) is typed
the math class is set to the class of the first
element of the \type{mathspec} table (\type{nothing} in this case).

\section{math-ini.lua}

All the information in the \LUA\ table in \filename{char-def.lua} by itself is
useless. We need to tell \CONTEXT\ how to use it. For the math mappings, this is
done in \filename{math-ini.lua}.

This file begins by defining names for
the different math classes:
\starttyping
local classes = {
    ord       =  0,  -- mathordcomm    
    op        =  1,  -- mathopcomm     
    bin       =  2,  -- mathbincomm    
    rel       =  3,  -- mathrelcomm    
    open      =  4,  -- mathopencomm   
    close     =  5,  -- mathclosecomm  
    punct     =  6,  -- mathpunctcomm  
    alpha     =  7,  -- mathalphacomm  
    accent    =  8,  -- class 0
    radical   =  9,
    xaccent   = 10,  -- class 3
    topaccent = 11,  -- class 0
    botaccent = 12,  -- class 0
    under     = 13,
    over      = 14,
    delimiter = 15,
    inner     =  0,  -- mathinnercomm  
    nothing   =  0,  -- mathnothingcomm
    choice    =  0,  -- mathchoicecomm 
    box       =  0,  -- mathboxcomm    
    limop     =  1,  -- mathlimopcomm  
    nolop     =  1,  -- mathnolopcomm  
}
\stoptyping

For each math class, this file has functions to return the
\LUATEX\ code to define the corresponding math characters. A few examples of
such functions:
\starttyping
local function delcode(target,family,slot)
    return format('\\Udelcode%s="%X "%X ',
                  target,family,slot)
end
local function mathchar(class,family,slot)
    return format('\\Umathchar "%X "%X "%X ',
                   class,family,slot)
end
local function mathaccent(class,family,slot)
    return format('\\Umathaccent "%X "%X "%X ',
                  0,family,slot) 
end
\stoptyping

\noindentation
Similar functions are there for defining math symbols, for example:
\starttyping
function setmathsymbol(name,class,family,slot)
  if class == classes.accent then
    texsprint(
        format("\\unexpanded\\xdef\\%s{%s}",
        name, 
        mathaccent(class,family,slot)))
  elseif class == classes.topaccent then
    texsprint(
        format("\\unexpanded\\xdef\\%s{%s}",
        name,
        mathtopaccent(class,family,slot)))
  elseif class == classes.botaccent then
    texsprint(
        format("\\unexpanded\\xdef\\%s{%s}",
        name,
        mathbotaccent(class,family,slot)))
  ...
end
\stoptyping

\type{math-ini.lua} then defines a \LUA\ function named \type{mathematics.define} that goes
through all the elements in the table in \filename{char-def.lua} and maps them
to the correct \LUATEX\ command.

These mappings are all that is needed to work with OpenType math fonts like
Cambria and Asana Math. \CONTEXT\ has predefined typescripts for Cambria; so, to
use Cambria you can just type
\starttyping
\usetypescript[cambria]
\setupbodyfont[cambria]
\stoptyping

There are no predefined typescripts for Asana Math, but defining one on our own
is not too hard. First, we need to define a math typescript:
\starttyping
\starttypescript [math] [asana] [name]
  \definefontsynonym
    [MathRoman] 
    [name:Asana-Math]
    [features=math\mathsizesuffix]
\stoptypescript
\stoptyping

\noindentation
The \type{features=math\mathsizesuffix} option activates the OpenType math
features. The rest of the typescript can be defined in the usual manner.

\starttyping
\starttypescript [asana]
  \definetypeface [asana] [rm] [serif] 
      [palatino] [default]
  \definetypeface [asana] [ss] [sans]  
      [modern]   [default] [rscale=1.075]
  \definetypeface [asana] [tt] [mono]  
      [modern]   [default] [rscale=1.075]
  \definetypeface [asana] [mm] [math]  
      [asana]    [default]
  \quittypescriptscanning
\stoptypescript
\stoptyping

\noindentation
To use this typescript, we need to type

\starttyping
\usetypescript[asana]
\setupbodyfont[asana]
\stoptyping

\section{math-map.lua and math-noa.lua}

So far, we have mapped Unicode input and macros to OpenType math fonts. However,
when using \TEX\ one expects traditional \TEX\ input to work. Thus, \type{$a$}
should typeset math italic $a$. No one is likely to type \type{0x1D44E}, even if
Unicode suggests that. The same is true for bold fonts. One expects \type{${\bf
a}$} to give bold ${\bf a}$, even if Unicode suggests that we should have typed
\type{0x1D41A}. 

To accommodate this, in math mode \CONTEXT\ maps upper and lower
case letters \type{A-Z}, \type{a-z},  digits \type{0-9}, and upper and lower
case Greek letters \type{α-ω}, \type{A-Ω} to the corresponding ranges in Unicode
math, depending on the current font style. These mappings are defined in
\filename{math-map.lua} file.

The mappings are defined using a \LUA\ table, which looks like this.
\starttyping
mathematics.alphabets = {
    regular = {
              tf = { ... },
              it = { ... },
              bi = { ... },
              bf = { ... },
    },
    sansserif = {
              tf = { ... },
              it = { ... },
              bi = { ... },
              bf = { ... },
    },
    monospaced = {
              tf = { ... },
    },
    blackboard = {
              tf = { ... },
    },
    fraktur = {
              tf = { ... },
              bf = { ... },
    },
    script  = {
              tf = { ... },
              bf = { ... },
    }
}
\stoptyping

Each of these subtables maps input letters to their corresponding Unicode
characters. These subtables look as follows.
\starttyping
regular = {
    ...
    it = {
      ucletters = 0x1D434,
      lcletters = { -- H
        [0x00061]=0x1D44E, [0x00062]=0x1D44F, 
        [0x00063]=0x1D450, [0x00064]=0x1D451, 
        [0x00065]=0x1D452, [0x00066]=0x1D453,
        [0x00067]=0x1D454, [0x00068]=0x0210E, 
        ... },
     symbols = {
        [0x0391]=0x1D6E2, [0x0392]=0x1D6E3, 
        [0x0393]=0x1D6E4, [0x0394]=0x1D6E5, 
        [0x0395]=0x1D6E6, [0x0396]=0x1D6E7,
        ... },
    },
},
\stoptyping

The line \type{ucletters = 0x1D434} tells \CONTEXT\ to map upper case letters to
Unicode characters starting from \type{0x1D434}. The line
\type{lcletters = {...}} tells \CONTEXT\ to map  \type{0x00061} to
\type{0x1D44E}, \type{0x00062}
to \type{0x1D44F}, as so on. For lower case letters, simply specifying
\type{lcletters = 0x1D44E} would not work because Unicode mathematical
italic small letters are not in contiguous slots. For example, the slot
\type{0x1D455} (which corresponds to lower case h) is empty; lower
case h should map to slot \type{0x0210E}.

Other subtables are filled in a
similar manner.

\filename{math-map.lua} also defines \LUA\ functions that use these tables to
remap characters on the fly. The actual transformation takes place in
\filename{math-noa.lua} which goes through the math noad list and carries out
the actual transformations according to the mappings in \filename{math-map.lua}.

\section{math-vfu.lua}

Using the above infrastructure, it is easy to use OpenType math fonts in
\CONTEXT. Unfortunately, at present there are only two Unicode math
fonts\Dash{}Cambria and Asana Math. OpenType math version of \TEX\ Gyre math fonts are
planned, but until they are developed, we need a way to use traditional \TEX\
fonts in \CONTEXT\ \MKIV. \CONTEXT\ creates virtual OpenType math fonts to use
traditional \TEX\ fonts. The mappings for creating the virtual font are in
\filename{math-vfu.lua}. Once a virtual OpenType math font is created, the above
infrastructure can be used to access the font.

First, \filename{math-vfu.lua} defines many encoding vectors that map Unicode
characters to glyph locations of the font. One such encoding vector is
\starttyping
fonts.enc.math["tex-mi"] = {
    [0x1D6E4] = 0x00, -- Gamma
    [0x1D6E5] = 0x01, -- Delta
    [0x1D6E9] = 0x02, -- Theta
    [0x1D6F3] = 0x02, -- varTheta (not in TeX)
    [0x1D6EC] = 0x03, -- Lambda
    [0x1D6EF] = 0x04, -- Xi
    [0x1D6F1] = 0x05, -- Pi
    [0x1D6F4] = 0x06, -- Sigma
    ...
}
\stoptyping

\noindentation
This tells that Unicode character \type{0x1D6E4} should be mapped to the font
glyph \type{0x00} and so on. A 
virtual font that associates such encoding vectors with
traditional \TEX\ fonts is created using
\starttyping
mathematics.make_font ( "lmroman10-math", {
  { name="lmroman10-regular.otf", 
    features="virtualmath", main=true },
  { name="lmmi10.tfm", vector="tex-mi", 
    skewchar=0x7F },
  { name="lmsy10.tfm", vector="tex-sy", 
    skewchar=0x30, parameters=true },
  { name="lmex10.tfm", vector="tex-ex", 
    extension=true },
  { name="msam10.tfm", vector="tex-ma" },
  { name="msbm10.tfm", vector="tex-mb" },
  { name="lmroman10-bold.otf", "tex-bf" } ,
  { name="lmmib10.tfm", vector="tex-bi", 
    skewchar=0x7F },
  { name="lmsans10-regular.otf", 
    vector="tex-ss", optional=true },
  { name="lmmono10-regular.otf", 
    vector="tex-tt", optional=true },
  { name="eufm10.tfm", vector="tex-fraktur", 
    optional=true },
  { name="eufb10.tfm", 
    vector="tex-fraktur-bold", optional=true },
} )
\stoptyping

\noindentation
This creates a virtual font \type{lmroman10-math} which takes bits and pieces
from various fonts. This virtual font can be used as follows.
\starttyping
\starttypescript [math] [modern]
  ...
  \definefontsynonym
    [LMMathRoman10-Regular]
    [LMMath10-Regular@lmroman10-math]
  ...
\stoptypescript
\stoptyping

\noindentation
The \type{@lmroman-math} in the name uses the above virtual
font. The \type{LMMathRoman10-Regular} font can be used to complete the
math typescript in the usual manner. 

\CONTEXT\ provides virtual OpenType math fonts for Latin Modern, Times
(\type{txfonts} and MathTime), Palatino (\type{pxfonts}), Iwona,
Lucida, and MathDesign (Charter, Garamond, and Utopia) fonts.

\section{Conclusion}

\CONTEXT\ now supports OpenType math fonts. In fact, even support for
traditional \TEX\ fonts now involves creating a virtual OpenType math font. Thus,
as far as \CONTEXT\ \MKIV\ is concerned, OpenType math fonts are the future.
However, the current implementation is still evolving, so some of the
implementation details described in this paper will likely change with time.

% context uses count1 instead of count0 for the page number
{\count0=\count1 \writelastpage{+1}}

\StopArticle
\stoptext

\startbibliography
\item[vieth] Ulrik Vieth, \quotation{\em OpenType Math Illuminated}, TUGboat
  30:1, 2009
\stopbibliography

