\enablemode[aditya]

% Preamble %{{{

\usemodule[tugboat]

\setupcolors[conversion=never] %otherwise \tracetablestrue gives greyscale

\setuplines[space=yes,
            before={\blank[tugblank]\bgroup\switchtobodyfont[small]\tt},
            after={\endgraf\egroup\blank[tugblank]}]

%\input tugboat.dates

%\def\startpage{501}
%\setfirstpageafter{abstracts}

% I wanted to check things without the manual corrections. Just remove
% \enablemode[aditya] from the top to get back the maual corrections.

\doifmodeelse{aditya}
 {\let\myparindent\relax % This interferes on my machine
  \let\mystrut\relax}%This is for testing
 {\def\myparindent{\leavevmode\kern20pt }
  \def\mystrut{\vrule width0pt height2.6ex depth0pt }}

%\abbreviation {HTML}  {Hyper Text Markup Language} 
\def\HTML{\acro{HTML}}

%LongLines % So that tables appear in the middle
           % No, this makes a mess of rules in the table.

\def\gettablebuffer
  {\dosingleargument\dogettablebuffer}

\def\dogettablebuffer[#1]%
  {\startlinecorrection
    \midaligned
      \bgroup
      \doifelsenothing{#1}
        {\getbuffer}%
        {\getbuffer[#1]}%
      \egroup
    \stoplinecorrection}

\def\typetablebuffer
  {\dosingleargument\dotypetablebuffer}

\def\dotypetablebuffer[#1]%
  {\blank[small]%
   \startpacked
    \typebuffer[#1]%
    \stoppacked
   \blank[small]}
    
\hyphenation
 {start-table
  start-tables
  start-tabulate
  start-line-table
  stop-table
  stop-tables
  stop-tabulate
  stop-line-table
  Wich-ura
  mne-mon-ic}


%}}}

% References {{{
% I could not get the bib module to print urls. Since urls are necessary, I
% reverted back to old style of placing urls as footnotes.

% \def\cite[#1]%
%   {\footnote{\mystrut \advance\rightskip by 0pt plus1pc
%    \hyphenpenalty=10000
%    \tt \from[#1]\par}}

\definereferenceformat[cite][left={[},right={]}]

\useURL[tablemacros][http://www.pctex.com/kb/47.html]
\useURL[beginnersmanual][http://www.pragma-ade.com/show-man-1.htm]
\useURL[tabledocumentation][http://store.pctexstore.com/maclimprom.html]
\useURL[core-tab][http://www.logosrl.it/context/modules/current/singles/core-tab_ebook.pdf]
\useURL[wiki-tables][http://wiki.contextgarden.net/Table]
\useURL[HansMaps][http://www.ntg.nl/maps/pdf/22_28.pdf]
\useURL[NaturalTables][http://www.pragma-ade.com/general/manuals/enattab.pdf]
\useURL[Willi][http://dl.contextgarden.net/myway/NaturalTables.pdf]
\useURL[wiki-comparison][http://wiki.contextgarden.net/Tables_Overview]


% }}}

\setuptyping[style={\tttf}]

\setvariables
  [tugboat]
  [type=article,
   columns=yes,
   grid=no]

\setupcolumns[balance=no]

% \setvariables
%   [tugboat]
%   [year=\tubyear,
%    volume=\tubvol,
%    number=\tubnum,
%    page=\startpage]

\setvariables
  [tugboat]
  [   title={\CONTEXT\ basics for users: Table macros},
   subtitle={Table Macros}, % Subtitle does not work :-(
     author={Aditya Mahajan},
    address={University of Michigan},
      email={adityam (at) umich dot edu},
  ]


\starttext
\StartArticle

\StartAbstract 
  \CONTEXT\ has four different table building macros. In the author's view the
  \type{table} macros, which are the oldest of the four, are the easiest to
  use for simple tables. This article gives a gentle introduction to these
  macros. 
\StopAbstract

\section Introduction %{{{1

Tables provide visual representation of information; understanding how to
build them in a markup language can be difficult and frustrating. To make
matters especially confusing for a new user, \CONTEXT\ provides four different
mechanisms to build tables |<|tables, tabulations, line tables, and natural
tables|>| and the user must decide which  one of the four to use. This can be
a difficult decision, because there is no \quotation{one size fits all}
solution. Each mechanism has its own strengths and weaknesses, and the best
match depends on the application.

Tables can get fairly complicated. They can have fancy horizontal and vertical
lines, colored rows, sophisticated \METAPOST\ backgrounds for cells, and/or
may need to be split across multiple pages. Some mechanisms are better
than others at handling such advanced features.

However, most users rarely need these features; for the most part, they
just want a simple, publication quality table. I think that out of the
four table building macros, the \mono{table} macros, which are based on
Michael Wichura's \TABLE\ package~\cite[tablemacros], are the easiest to
use for simple tables.

\myparindent
I should note that this mechanism is no longer actively developed, and the more modern
natural table macros are the unofficially recommended choice.  However, I find
natural tables to be a bit too verbose. The \mono{table} macros take care of
the {\em simple} tables. If you need to typeset {\em complicated} tables, you
need to look into all the mechanisms and decide which one suits your needs
best.

% First lets look
% at some of the limitations of \mono{table} macros:
% \startitemize[n]
%   \item No automatic calculation of the width of paragraph columns.
%   \item It is not easy to separate out content from formatting information.
%   \item No real support for cells spanning multiple rows. 
%   \item Table cannot split horizontally across pages (it can split vertically
%     across pages).
% \stopitemize
% 
% Note that these are the limitations of the \mono{table} macros, and not of
% \CONTEXT\ table mechanisms. The other mechanisms can be used to get these
% features.

This article explains the basics of the \mono{table} macros. It is cumbersome
to separate out the features of Michael Wichura's \TABLE\ package, and those
added or adapted by Hans Hagen in the course of integrating with
\CONTEXT. For simplicity of exposition and with sincere
apologies to Michael, I am going to refer to all these features as features of
\CONTEXT's \mono{table} macros. 

\section How to build tables %{{{1

The basic structure of a table built with the \mono{table} macros looks
like this:

\startlines
\tex{starttable}[\slanted{preamble}]
 \tex{NC} \dots \tex{NC} \tex{AR} \% first row
     \dots         \% middle rows
 \tex{NC} \dots \tex{NC} \tex{AR} \% last row
\tex{stoptable}
\stoplines

The \slanted{preamble} indicates the number and formatting of the
columns. The column formatting is defined by ``column specifiers'',
separated by vertical bars~\type{|}; we also put bars at the beginning
and end of the preamble.  These \type{|} characters do {\em not} indicate
vertical lines, as they do in \LATEX; they simply separate columns.

Each column specifier consists of keys,
which are single characters, % xx true? yes. Although keys can take parameters
and the key combination indicates the formatting for that column.  
The number of columns is simply the number of
column specifiers (that is, the number of \type{|}'s minus one).

The body of the table can contain an arbitrary number of rows. Each row has
the form

\startlines\sl
  \tex{NC} cell 1 \tex{NC} \dots \tex{NC} cell $n$ \tex{NC} \tex{AR}
\stoplines

The \tex{NC} (mnemonic: new column) control sequence is the usual column
separator, and \tex{AR} (mnemonic: automatic row) is the usual row
separator. Each column separator corresponds to one~\type{|} in the
preamble.

% It is also possible to use the place the preamble in square brackets instead
% of curly brackets, but that makes it cumbersome to get the numeric columns
% which use square brackets as part of their specification. So in this article,
% I will only use a preamble with curly brackets.

\startbuffer[pre-1]
\starttable[|l|c|r|]
\stopbuffer

\startbuffer[body]
  \NC left \NC center \NC right \NC \AR
  \NC 23   \NC 45     \NC 67    \NC \AR
\stoptable
\stopbuffer

%  \NC 9    \NC 9      \NC 9     \NC \AR

Let's consider an example. Suppose we want to produce the following table:
\gettablebuffer[pre-1,body]

Each row consists of three columns, with the first one left aligned, the second
center aligned, and the last right aligned. The preamble for this table is
\type{[|l|c|r|]}. Here \type{l} is an abbreviation for
left,\footnote{\mystrut \type{l} is
actually {\em ell}.  In the fonts used for this article, it is difficult to
distinguish \type{l} (ell) from \type{1} (one). To make things easier for the
reader, I will not use the digit \type{1} (one) in any example.} \type{c} for
center, and \type{r} for right. Each column is as wide as necessary to
accommodate the width of its contents.

\myparindent The input for the above table is:

\typetablebuffer[pre-1,body]

The ``extra'' \tex{NC}'s at the end of each row are required. Omitting
them can lead to hard-to-detect problems.

%Notice that the preamble contains the
%vertical bar~\type{|} even though the table does not have vertical lines. This
%is different from \LATEX\ where~\type{|} in the preamble of \mono{tabular}
%environment means vertical lines. If you want vertical lines you need to use
%\tex{VL} instead of \tex{NC} as the column separator. 

\section Specifying column widths %{{{1

By default, each column is wide enough to accommodate the largest entry in that
column. This may not always be desired. If you want to ensure that each column
has a certain {\em minimum} width, use \type{w(width)} in the column
specifier.  For example, if we want the middle column in the above table to be
at least 4\,cm wide, we can specify the preamble as
\startbuffer[pre-2]
  \starttable[|l|cw(4cm)|r|]
\stopbuffer

\typetablebuffer[pre-2]

\noindent which gives

\gettablebuffer[pre-2,body]

On the other hand, if you want to specify a certain {\em maximum} width of a
column, use the \type{p(width)} in the column specifier. Such columns are
called paragraph columns. For example, suppose we want a column to be
5\,cm wide:

\startbuffer[pre-1]
\starttable[|l|lp(5cm)|]
\stopbuffer

\startbuffer[body]
  \NC TUGboat
  \NC The TUGboat journal is a unique benefit of joining TUG. It is
  currently published three times a year \dots
  \NC \AR
\stoptable
\stopbuffer

\startbuffer[body-condensed]
  \NC TUGboat
  \NC The TUGboat journal ...
  \NC \AR
\stoptable
\stopbuffer

\gettablebuffer[pre-1,body]

\noindent We can use the following:

\typetablebuffer[pre-1,body-condensed]

This \type{lp(5cm)} key combination gives us left-justified text,
i.e., ragged right.  You can also use \type{rp(5cm)} to get
right-justified (ragged left) text, and \type{cp(5cm)} to get centered
text.

However, in most cases we want such paragraph columns to be
justified. To achieve this we can use \type{xp(5cm)}, which gives:

% cheat to avoid overfull hbox
\startbuffer[pre-2]
\starttable[|l|xp(5.1cm)|]
\stopbuffer

\gettablebuffer[pre-2,body]

\blank
\section Horizontal and vertical lines %{{{1

To get horizontal lines that span the width of the entire table, you can use
\tex{HL} between the rows where a line is desired.

By default, the width of the line (a.k.a.\ ``rule'') is 0.4\,pt\Dash \TEX's
default for lines.
The linewidth can be controlled by the \mono{rulethickness} option of
\tex{setuptables}. For example, if you want 2\,pt thick lines, use
\type{\setuptables[rulethickness=2pt]}. 

The width of a specific line can be increased using an optional argument to
\tex{HL}. This argument must be an integer, and it increases the width of the
current line by that factor. For example:

\startbuffer
\setuptables[rulethickness=0.03em]
\starttable[|l|l|r|]
  \HL[3]
  \NC Animal \NC Desc \NC Cost (\$) \NC \AR
  \HL
  \NC Gnat \NC per gram \NC 13.65 \NC \AR
  \NC Emu  \NC stuffed  \NC 33.33 \NC \AR
  \NC Gnu  \NC stuffed  \NC 92.50 \NC \AR
  \HL[3]
\stoptable
\stopbuffer

\typetablebuffer

% not needed? Needed for tracing example.
%   \NC Emu  \NC stuffed  \NC 33.33 \NC \AR


\noindent gives

{\gettablebuffer}

Here the first and last lines are $3*0.03\,{\rm em} = 0.09\,{\rm em}$
thick, while the middle line is the default 0.03\,em.  Notice that the
height of the rows around the horizontal lines is automatically
adjusted\Dash that's the \type{A} in \tex{AR}.

Other row specifiers allow manual adjustment of space: \tex{NR} (new
row), \tex{SR} (single row), \tex{FR} (first row), \tex{MR} (middle
row), and \tex{LR} (last row). Depending on the surrounding rows, \type{\AR}
is converted into one of these row specifiers.

One can use \type{\tracetablestrue} to see what \CONTEXT\ is doing behind
the scenes. If we add \type{\tracetablestrue} before calling the above table,
we get:

{\tracetablestrue\gettablebuffer}

The first row was surrounded by two horizontal lines, so the \type{\AR} in
the first row was changed into \type{\SR} (single row). This introduces some
space above and below the row, so that the row is not too close to the
horizontal lines.

The second row has a horizontal line above it, so the \type{\AR} 
was changed into \type{\FR} (first row). This adjusts the space above
the row to provide some distance from the horizontal line.

The third row does not have a horizontal line above or below it, so the
\type{\AR} is changed into \type{\MR} (middle row). This adds normal
inter|-|row space above and below the row.

The last row had a horizontal line below it, so the \type{\AR} was changed
into \type{\LR} (last row). This adjusts the space below the row to provide
some distance from the horizontal line.

The \type{\NR} command sequence does not adjust space at all. It should be
used only when such tight spacing is required.

If you feel that \CONTEXT\ has made a wrong choice, you can use the desired
row separator instead of \type{\AR} to end that particular row in the table.

% Division lines. Talk about them in the next issue.

Vertical lines are rarely considered good typography. However, if you insist,
use \tex{VL} as the column separator instead of
\tex{NC}. For example, this input:

\startbuffer
\starttable[|l|cw(4cm)|r|]
  \HL
  \VL left \VL center \VL right \VL \AR
  \HL
  \VL 23   \VL 45     \VL 67    \VL \AR
  \HL
\stoptable
\stopbuffer

\typetablebuffer

%  \HL
%  \VL 9    \VL 9      \VL 9     \VL \AR

\noindent gives

\gettablebuffer

This also shows that with the \type{w} key, the width of the middle column was
increased, in this case with centered text.

\section Documentation %{{{1

So far I have shown some very basic features of the \mono{table} macros. A few
more details of these macros are documented in the beginner's
manual~\cite[beginnersmanual].  However, the manual does not list all features
of these macros. Documentation of the features inherited from the
\TABLE\ package is available in its own manual, which is sold by
PC\TEX~\cite[tabledocumentation].

\myparindent
Unfortunately, there is no real documentation of the enhancements,
especially with regards to
color, done by \CONTEXT. There are a few examples in the source file
\filename{core-tab.tex}~\cite[core-tab], and a few more on the \CONTEXT\
wiki~\cite[wiki-tables].

\myparindent
So, in the next issue of this series, I will explain some additional features of
\mono{table} macros, such as specifying the font style and color of each column,
spanning multiple rows and columns, controlling the space between the columns,
and splitting tables across pages.

\section Other mechanisms %{{{1

To conclude this installment, here is a brief overview of the other
table-building mechanisms mentioned earlier.

The tabulate macros were added for building running text aligned blocks like
formula legends and facts. They are capable of automatic width calculation of
paragraph columns (similar to the \filename{tabularx} package in \LATEX) and
splitting across pages. They are built on the same underlying principle as the
\mono{table} macros, and in due time will be backward compatible with them.
However, at present they do not support vertical rules and colors.
Documentation and examples of tabulations are given in a
{\acro{\sl MAPS}} article~\cite[HansMaps] by Hans Hagen.

\myparindent
Line tables are experimental macros for building large tables that can be
split horizontally and vertically. It is possible to repeat table header lines
and entry columns. Unfortunately, these macros are largely undocumented.

Natural tables are the most configurable table macros. Their syntax is
inspired from \HTML\ tables. It is possible to change the color, background,
and style of a single cell, row, or column, and of odd or even rows and
columns. It is also easy to use \METAPOST\ backgrounds. Again, there is no detailed
documentation, but many examples are present in Hans's
example document~\cite[NaturalTables]; another
interesting example is Willi Egger's MyWay~\cite[Willi].

\myparindent
Finally, the pros and cons of each of these mechanisms is summarized in
an article in the \CONTEXT\ Garden wiki~\cite[wiki-comparison].

\subject Bibliography

% Quick and dirty.

\startitemize[n][left={[},right={]}, width=1.5em, stopper=]
  \item[tablemacros] Michael Wichura: The \TABLE\ Macro Package.
    {\tt\from[tablemacros]}.
  \item[beginnersmanual] Hans Hagen: \CONTEXT\ an excursion.
    {\tt\from[beginnersmanual]}.
  \item[tabledocumentation] Michael Wichura: \PICTEX\ and \TABLE\ manual.
    {\tt\from[tabledocumentation]}.
  \item[core-tab] Hans Hagen: \CONTEXT\ core macros\Dash \TABLE\ embedding.
    \hyphenpenalty=10000 {\tt\from[core-tab]}.
  \item[wiki-tables] \CONTEXT\ garden\Dash Table. \crlf {\tt\from[wiki-tables]}.
  \item[HansMaps] Hans Hagen: Tabulating in \CONTEXT\Dash text flow tables.
    {\it NTG MAPS}, Spring 1999. {\tt\from[HansMaps]}.
  \item[NaturalTables] Hans Hagen: Natural tables in \CONTEXT. 
    {\tt\from[NaturalTables]}.
  \item[Willi] Willi Egger: My Way\Dash Use of natural table environment. 
    {\tt\from[Willi]}.
  \item[wiki-comparison] \CONTEXT\ garden\Dash Tables Overview. \crlf
    {\tt\from[wiki-comparison]}.
\stopitemize


% }}}

%\setlastpage{1}

\StopArticle
\stoptext
