
\usemodule[tugboat]

\input tugboat.dates
\def\tubpage{1001}

\setvariables
  [tugboat]
  [type=article,
   columns=yes,
   grid=no]

 \setvariables
    [tugboat]
    [year=\tubyear,
     volume=\tubvol,
     number=\tubnum,
     page=\tubpage]

\setvariables
  [tugboat]
  [   title={\CONTEXT\ basics for users: Indentations},
     author={Aditya Mahajan},
    address={University of Michigan},
      email={adityam (at) umich dot edu},
  ]

% Typo in core-spa.tex

\unprotect

\defineindentingmethod [\v!reset]  {\indentfirstparagraphtrue
                                    \parindent\zeropoint
                                    \chardef\indentingtoggle\zerocount}

\protect

\hyphenation
  {indentnext
   setupindenting}

\starttext
\StartArticle 

\StartAbstract
  \CONTEXT's indentation mechanism can be a bit confusing. This
  article explains why \CONTEXT\ indentation works the way it does and
  how to set up indentation to achieve desired behaviour.
\StopAbstract

\section Introduction %{{{1

In plain \TEX, controlling indentation is simple: The user sets a value for
\tex{parindent}, and each new paragraph is indented by that value, unless
explicitly begun with \tex{noindent}. Environments can provide a
\tex{noindent} at the end of their definitions, and if the user wants to
overrule that, he can add an explicit \tex{indent} at the beginning of the next
paragraph. For the most part, \LATEX\ follows the same convention.

So,
understanding indentation in plain \TEX\ and \LATEX\ boils down to
this: set a
value for \tex{parindent}, and start a new paragraph (i.e., leave an empty
line) whenever you want indentation. For example, \LATEX\ usually does not
indent the first line after a sectioning command. If you want to indent the
first line after a sectioning command you use the \type{indentfirst} package
(which is part of the required \LATEX\ bundle). If you want to indent the
paragraph after an environment, you leave a blank line after the end of the
environment; if you do not want to indent after the end of an environment, you
do not leave a blank line. It takes a while to get used to, but the rules are
easy to remember and eventually you do not need to even think about
indentation; it becomes a matter of habit.

Indentations in \CONTEXT\ are a bit different; and sometimes difficult
to understand. In this article I hope to explain how \CONTEXT\ does
indentations.  First, let's understand why \CONTEXT\ does indentations
differently; why does it not simply follow the time|-|tested approach
of plain \TEX\ and \LATEX? The way I understand it, the reason is that Hans
Hagen, the author of \CONTEXT, prefers 
{\em spaced out} markup\Dash surrounding each environment by empty
lines\Dash which
makes it
easy to see where an environment starts and ends while reading the
source file. However, this style means that the \quotation{indentation
  after empty lines} paradigm of plain \TEX\ and \LATEX\ cannot be
used for indentations. So, \CONTEXT\ provides an alternative. As
with other things in \CONTEXT, this alternative is consistent and easy
to configure; but if you are used to other \TEX\ formats it takes some time to
get comfortable with it.

\section The basics %{{{1

Indentation involves two things: when to indent, and how much to indent. In
\CONTEXT, these can be specified using \tex{setupindenting[...]}. There
are two types of keys for this command:
\startitemize[n,intro]
\item To specify {\em when to indent}:
  \type{never} or \type{always}, (equivalently, \type{no} or \type{yes}),
  \type{odd} or \type{even}, and \type{first} or \type{next}.

\item To specify {\em how much to indent}:
  \type{none}, \type{small}, \type{medium}, \type{big}, or a specific
  dimension. \type{small} corresponds to 1em, \type{medium} to 1.5em, and
  \type{big} to 2em.

\stopitemize

Suppose we want to indent all paragraphs by 20pt, which is the
convention followed by this journal: we can use
\tex{setupindenting[20pt, yes]}. This is what one typically uses in a
document. The other keys are needed only for special cases (like
typesetting quotations and verses), and we will not talk about them in
this article.



\tex{setupindenting} does not take care of indenting after environments,
such as \CONTEXT's itemizes (approximately \LATEX\ \type{itemize} and
\type{enumerate}), enumerations ($\approx$ \LATEX\ \type{theorem}),
definitions ($\approx$ \LATEX\ \type{subparagraph}), formulas and
floats. It also does not take care of indenting after heads such as
chapters, sections, and subsections.

The setup command of these environments provides an \type{indentnext} key to
configure the indentation behaviour after the environment. The
\type{indentnext} key can take one of three values: \type{yes}, \type{no}, and
\type{auto}. If \type{indentnext=yes} then the paragraph after the end of the
environment is always indented; if \type{indentnext=no} then the next paragraph
is never indented; if \type{indentnext=auto} then the next paragraph is
indented only if there is a blank space after the environment. Setting
\type{indentnext=auto} is equivalent to the default plain \TEX\ and \LATEX\
behaviour. 

Let us provide a couple of examples. Suppose we do not want paragraphs after
itemize to be indented; then we can say

\startbuffer
\setupitemgroup[itemize][indentnext=no]
\stopbuffer

\typebuffer

\startmode[check]
\start
\getbuffer

\startitemize[n]
\item one
\item two
\stopitemize

This paragraph should not be indented
\stop
\stopmode

\noindentation If we want paragraphs after section heads to be indented, we can say

\startbuffer
\setuphead[section][indentnext=yes]
\stopbuffer

\typebuffer

\startmode[check]
\start
\getbuffer

\section{Indented section}

This should be indented

\stop
\stopmode

\noindentation If we want the paragraphs after all sectioning heads to be
indented, we can say

\startbuffer
\setupheads[indentnext=yes]
\stopbuffer

\typebuffer

\noindentation If we want paragraphs after formulas to be indented only if we
leave a blank space after them, we can say

\startbuffer
\setupformulas[indentnext=auto]
\stopbuffer

\typebuffer

\startmode[check]
\start
\getbuffer
\startformula
  E = mc^2
\stopformula
This should not be indented

\startformula
 a^2 + b^2 = c^2 
\stopformula

This should be indented
\stop
\stopmode

There is one case that is not taken care of by \tex{setupindenting} and the
\type{indentnext} key: indentation of paragraphs inside multi|-|paragraph
environments such as itemizes, descriptions, and enumerations. By default
\CONTEXT\ does not indent such paragraphs. The setup commands of these
environments provides an \type{indenting} key to configure the indentation
behaviour of paragraphs inside these environments. This key takes the same
values as the arguments of \tex{setupindenting} command. For example, if we set
\startbuffer[itemize]
\setupitemgroup[itemize][indenting={40pt,yes}]
\stopbuffer

\typebuffer[itemize]

\noindentation then the following (\type{|-|} in \CONTEXT\
indicates a hyphen after which further hyphenation is allowed;
see \type|http://www.logosrl.it/context/modules/|\hfil\break
\type|current/singles/lang-mis_ebook.pdf| for more):

\startbuffer[example]
\startitemize
\item This is an example of a multi|-|paragraph 
      item inside an itemize environment.

      This second paragraph is indented by 40pt
      (double the normal indentation).
\stopitemize
\stopbuffer

\typebuffer[example]

\noindentation gives

\blackrule[width=\textwidth,height=0.6pt]

\start
\getbuffer[itemize,example]
\stop
\blackrule[width=\textwidth,height=0.6pt]


\starthiding
\section Indentation after floats

This is a mess. This will take a lot of time to explain.
\stophiding

\section Manual indentation %{{{1

No matter how careful we are with all the settings, there are some cases which
cannot be taken care of by automatic indentation, and we have to tell \CONTEXT\
how to indent. Plain \TEX\ (and \LATEX) provide the \tex{indent} and \tex{noindent}
commands for explicitly indenting and preventing indenting of a paragraph. These
commands are defined in \CONTEXT\ but are not hooked into the \CONTEXT\
indentation mechanism. Instead, \CONTEXT\ provides \hbox{\tex{indentation}} and
\tex{noindentation} which achieve the same effect.

Notice that in this article, paragraphs after itemize are not indented. This is
because the itemize environment has been set up (according to the general
style of the journal) as
\starttyping
\setupitemgroup [itemize] [indentnext=no]
\stoptyping

Now, if we want to indent after an itemize, starting the next paragraph
with \tex{indent} does not work; to get indentation we have to say
\tex{indentation}. For example,

\startbuffer
\startitemize[n]
\item A dummy list
\item To check \tex{indent}
\stopitemize

\indent This paragraph is not indented. The 
\tex{indent} command does not work here.

\startitemize[n]
\item Another dummy list
\item To check \tex{indentation}
\stopitemize

\indentation This paragraph will be indented
because we used \tex{indentation} instead.
\stopbuffer

\typebuffer

\noindentation gives

\blackrule[width=\textwidth,height=0.6pt]

\getbuffer

\blackrule[width=\textwidth,height=0.6pt]

\section Beware of typos %{{{1

\CONTEXT\ defines two more commands not commonly used:
\tex{indenting} and \tex{noindenting}.  \tex{indenting} is similar to
\tex{setupindenting} and is provided for backward compatibility;
\tex{noindenting} is equivalent to
\tex{setupindenting[no]}. Unfortunately, these commands sound very
similar to \tex{indentation} and \tex{noindentation}, thus can be easily 
used by mistake. If you happen to write \tex{noindenting} instead of
\tex{noindentation} in a document, it can take a while to debug. I have
been bitten by this mistake once too often so I have invented a mnemonic
to avoid it: 
\startnarrower
  {\em shun (-tion) the -ing}|<|use indentation 
\stopnarrower 
\noindentation I admit, this is not a clever mnemonic, but it has
saved me quite a few trips to the manual.

\section Conclusion %{{{1

This article explained how indentation works in \CONTEXT. By default, \CONTEXT\
does not enable {\em any} indentation; the user is expected to set up
indentation as he wants. I hope this article has helped to illustrate how to
set up indentation for different environments. 



% Odd, Even %{{{1
\startmode[check]
\startbuffer
Par 1 \par Par 2 \par Par 3 \par
\stopbuffer

{\setupindenting[reset,20pt,yes,first] \getbuffer}

{\setupindenting[reset,20pt,no,first] \getbuffer}

{\setupindenting[reset,20pt,yes,next] \getbuffer}

{\setupindenting[reset,20pt,no,next] \getbuffer}

{\setupindenting[reset,20pt,yes,odd] \getbuffer}

{\setupindenting[reset,20pt,no,odd] \getbuffer}

{\setupindenting[reset,20pt,yes,even] \getbuffer}

{\setupindenting[reset,20pt,no,even] \getbuffer}

\stopmode

% }}}

\StopArticle
\stoptext
