%D \module
%D   [      file=t-tugboat
%D        version=$Id: t-tugboat.tex 85 2011-05-23 16:06:42Z karl $
%D          title=\CONTEXT\ Style File,
%D       subtitle=\TUGBOAT\ base style,
%D         author=Aditya Mahajan,
%D           date=\currentdate,
%D      copyright=Public Domain]

%D This file is derived from \filename{s-tug-02} written by Hans Hagen
%D and Steve Grathwohl. Karl Berry asked me for some changes in the
%D \TUGBOAT\ style files. The original file had an option of typesetting
%D on a grid. That was too clever for me to understand, and hence to
%D modify. Since \TUGBOAT\ usually does not typeset on a grid, I redid the
%D style, borrowing parts from \filename{s-tug-02} and redoing some
%D parts from scratch.

%D \section Variables %<<<1
%D
%D We store the information about the article in variables.

\setvariables
  [tugboat]
  [type=article,
   columns=no,
   grid=yes
   ]

\setvariables
  [tugboat]
  [year=1900,
   volume=0,
   number=0,
   page=1001]

\setvariables
  [tugboat]
  [title=ConTeXt style for TUGboat,
   subtitle=,
   keywords=,
   author=T. Boat,
   address=Pragmatically Advanced tugboats \\
           314 Pi Ave. \\
           8061GH Hasselt NL,
   email={tugboat@tug.org}]

%D \section Font Setup %<<<1
%D 
%D \TUGBOAT\ uses slightly different interline space than the default.
%D So we change the interline space.

\definebodyfontenvironment    [8pt] [interlinespace=9.5pt, big=9pt,  small=7pt]
\definebodyfontenvironment    [9pt] [interlinespace=11pt, big=10pt,  small=8pt]
\definebodyfontenvironment   [10pt] [interlinespace=12pt, big=12pt,  small=9pt]
\definebodyfontenvironment   [12pt] [interlinespace=14pt,big=14.4pt,small=10pt]
\definebodyfontenvironment [14.4pt] [interlinespace=18pt,big=14.4pt,small=12pt]

%D \TUGBOAT\ uses Computer Modern fonts, and \CONTEXT\ uses Latin Modern
%D by default. So, we just specify the font size.

\setupbodyfont [10pt]

% Italic rather than slanted for emphasis.
\setupbodyfontenvironment[default][em=italic]

% Break after these chars in urls, not before.
\sethyphenatedurlafter /
\sethyphenatedurlafter .
\sethyphenatedurlafter _

%D \section Layout Setup %<<<1
%D
%D The original layout used in the \LATEX\ style for \TUGBOAT\ is a bit
%D ambiguous. It uses low|-|level \TEX\ syntax, rather than changing the
%D layout in a human understandable way (for example, by using the
%D \mono{geometry} package. I have tried to translate it to \CONTEXT\ as
%D far as I understand.

\setuppapersize[letter][letter]

\setuplayout
  [topspace=3.8pc,% was 3.5pc
   header=1pc,
   headerdistance=1.5pc,
   height=middle,
   footerdistance=2pc,
   footer=1pc,
   bottomspace=3pc,
   %
   backspace=6pc,
   width=middle,
   cutspace=6pc,
   %
   margin=4pc,
   margindistance=1pc,
  ]

\setupcolumns[distance=1.5pc]

\setuppagenumbering
  [location=,
   alternative=doublesided]

%D In \TUGBOAT\ different articles are glued together to form the final
%D journal, so we do not want each article to occupy even number of
%D pages.

\installpagebreakhandler {last} {}

%D \section Indentation %<<<1
%D
%D We use automatic indentation control, that is: no
%D indentation after titles and skips.

\setupindenting[20pt,yes]

%D We do not want indentation after lists.

\setupenumerations [indentnext=no]

\setupdescriptions [indentnext=no]

%D \section Itemize 
%D
%D And these. We typeset itemizations ragged right.

\setupitemgroup
  [itemize]
  [indentnext=no,
   align=right]

%D We align them at the paragraph indentation and
%D pack them by default.

\setupitemgroup
  [itemize]
  [each]
  [margin=1pc,
   width=1em,
   distance=0pt]

\setupitemgroup
  [itemize]
  [1]
  [packed]

%D \section Section Headings %<<<1
%D
%D We follow the \TUGBOAT\ style for sections. I do not know if
%D \type{align=right} also disables hyphenation. Lets wait and see on
%D this. Rest all is straight forward. It took me a while to realize
%D that in \LATEX\ \type{\@startsection} the absolute value of before
%D skip (fourth argument) is important and not the sign.

\setuphead
  [section,subsection, subsubsection,
   subject, subsubject, subsubsubject]
  [style=bold,
   align=right,
   before={\blank[8pt]}, 
   after={\blank[4pt]}]

%D \section Spacing <<<1
%D
%D We define a logical skip. This is equal to the \tex{topsep} in latex,
%D and most environments should have this skip.

\defineblank[tugblank][3pt]
\setupblank[3pt]


\setupitemize[1][before={\blank[tugblank]},after={\blank[tugblank]},
                 inbetween={\blank[tugblank]}]

\setuplines[before={\blank[tugblank]},after={\blank[tugblank]}, 
            inbetween={\blank[tugblank]}]                       

%D \section Typing %<<<1
%D
%D \TUGBOAT\ uses a smaller font size for verbatim typesetting.

\setuptyping
  [option=none,
   before={\blank[tugblank]\switchtobodyfont[small]},
   after={\blank[tugblank]}]

%D \section Footnotes %<<<1
%D
%D Not entirely a la \TUGBOAT:

\setupfootnotes
  [bodyfont=8pt,
   location=columns,
   rule={\hrule width 5pc height .4pt depth 0pt\relax \kern \strutdepth}]

\setupnotation[footnote]
  [location=joinedup,
   width=fit,
   headstyle=normal,
   distance=.5em]

%>>>
%D \section List %<<<
%D
%D We define a standard description and enumeration
%D environment.

\definedescription
  [description]
  [location=hanging,
   width=broad,
   before={\blank[tugblank]},
   after={\blank[tugblank]}]

\defineenumeration
  [enumeration]
  [location=hanging,
   width=broad,
   before={\blank[tugblank]},
   after={\blank[tugblank]}]


% >>>
%D \section References %<<<
%D
%D The bib does not handle urls nicely. So we provide a stop gap solution.

\definereferenceformat[cite][left={[},right={]}]
\defineitemgroup [bibliography] [levels=1]
\setupitemgroup  [bibliography] 
                 [symbol=n,
                  left={[},
                  right={]},
                  width=1.5em,
                  stopper=,
                  itemalign=flushright,
                  inbetween={\blank[small]}]

%D Instead of color, we use weighted gray scales:
\setupcolors
  [conversion=always]

%D English it is.
\mainlanguage
  [en]

%D We define some logical skips

\defineblank [tugbefore]   [big]
\defineblank [tuginbetween][big]
\defineblank [tugafter]    [tugbefore]

%D Some real macros: <<<

\def\StartAbstract
  {\dostartbuffer[abstract][StartAbstract][StopAbstract]}

\startsetups tugboat:abstract:setup

  \setuptolerance
    [horizontal, tolerant]

  \setupnarrower
    [before={\blank[.5\baselineskip]},
    after={\blank[10pt plus4pt minus4pt]},
    middle=4.875pc]

\stopsetups

%D Headers and footers are different for normal issues
%D and proceedings.

\startsetups tugboat:banner:text:article

  {\sl TUGboat},\space
  Volume \getvariable{tugboat}{volume}\space
  (\getvariable{tugboat}{year}),\space
  No.\space\getvariable{tugboat}{number}

\stopsetups

\def\postissno{Proceedings of the \tubyear\ Annual Meeting}

\startsetups tugboat:banner:text:proceedings

  \setups{tugboat:banner:text:article}
  \thinspace---\thinspace
  \postissno

\stopsetups

\startsetups tugboat:banner:setup:article

  \setupheadertexts
    [\setups{tugboat:banner:text:article}]
    [\pagenumber]

% no footer in regular articles
%  \setupfootertexts
%    [][\getvariable{tugboat}{author}]
%    [\getvariable{tugboat}{title}][]

\stopsetups

\startsetups tugboat:banner:setup:proceedings

  \setupheadertexts
    [][\getvariable{tugboat}{title}]
    [\getvariable{tugboat}{author}][]

  \setupfootertexts
    [\setups{tugboat:banner:text:proceedings}]
    [\pagenumber]

\stopsetups

%D article is default so,

\setups{tugboat:banner:setup:article}

%D It all starts here:

\def\StartArticle{\setups{tugboat:\getvariable{tugboat}{type}:start}}
\def\StopArticle {\setups{tugboat:article:stop}}

\startsetups tugboat:grid:setup:settings:yes

  \setupblank
    [line]

  \defineblank [tugbefore]    [halfline]
  \defineblank [tuginbetween] [halfline]

\stopsetups

\startsetups tugboat:grid:setup:no

%   \setupblank
%     [halfline]

%   \defineblank [medium]       [halfline]
%   \defineblank [tugbefore]    [halfline]
%   \defineblank [tuginbetween] [halfline]

\stopsetups

\startsetups tugboat:introduction:article

%   \blank[halfline]

  \start
    \def\\{\unskip\space\&\space\ignorespaces}
    \hbox{\indent\getvariable{tugboat}{author}}
    \par
  \stop


\stopsetups

\startsetups tugboat:introduction:proceedings

  \blank[20pt]

  \start
    \switchtobodyfont[12pt]
    \def\\{\unskip\space\&\space\ignorespaces}
    \getvariable{tugboat}{author}
    \par
  \stop

    \start
    \switchtobodyfont[9pt]
    \def\\{\unskip,\space\ignorespaces}
    \getvariable{tugboat}{address}
    \par
    {\tt\getvariable{tugboat}{email}}
    \par
  \stop


\stopsetups

\startsetups tugboat:article:start

  \starttext

  \setups{tugboat:columns:\getvariable{tugboat}{columns}}
  \setups{tugboat:banner:setup:\getvariable{tugboat}{type}}

  \doif{\getvariable{tugboat}{columns}}{yes}{\startcolumns}


  % AM: Why set these again?
  % \setupheadertexts
  %   [\setups{tugboat:banner:text:article}]
  %   [pagenumber]

  % \setuppagenumber
  %   [number=\getvariable{tugboat}{page}]
  \setcounter[userpage][\getvariable{tugboat}{page}]

  % \setuplayout
  %   [grid=\getvariable{tugboat}{grid}]

  % % instead of \startmode   [*grid] ... \stopmode
  % % instead of \startnotmode[*grid] ... \stopnotmode

  \setups{tugboat:grid:setup:\getvariable{tugboat}{grid}}

  \snaptogrid \vbox \bgroup

    \forgetall
\hrule height .6pt
\blank[halfline]
    \start
      \let\\=\par
      {\bf\getvariable{tugboat}{title}}
      \par
      \blank[halfline]
      \hskip20pt\getvariable{tugboat}{author}
    \stop

%     \setups{tugboat:introduction:\getvariable{tugboat}{type}}

    \blank[line]

    \doiftext{\getbuffer[abstract]}
    {\let\\=\endgraf
     \setups{tugboat:abstract:setup}
     \subject{Abstract}
     \getbuffer[abstract]}
  \egroup
\stopsetups

\startsetups tugboat:proceedings:start

  \starttext

  \setups{tugboat:columns:\getvariable{tugboat}{columns}}
  \setups{tugboat:banner:setup:\getvariable{tugboat}{type}}

  \setupheader
    [state=empty]

  \setuppagenumber
    [number=\getvariable{tugboat}{page}]

  \setuplayout
    [grid=\getvariable{tugboat}{grid}]

  % instead of \startmode   [*grid] ... \stopmode
  % instead of \startnotmode[*grid] ... \stopnotmode

  \setups{tugboat:grid:setup:\systemmodevalue{grid}{yes}{no}}

  \snaptogrid \vbox \bgroup

    \forgetall

    \start
      \switchtobodyfont[14.4pt]
      \let\\=\par
      \getvariable{tugboat}{title}
      \par
    \stop

    \setups{tugboat:introduction:\getvariable{tugboat}{type}}

    \blank[9pt plus3pt minus3pt]

    \let\\=\par
    \setups{tugboat:abstract:setup}
    \midaligned{\bf Abstract}

    \startnarrower[middle]
       \getbuffer[abstract]
    \stopnarrower

  %  \blank[10pt plus4pt minus4pt]

  \egroup

  \doif{\getvariable{tugboat}{columns}}{yes}{\startcolumns}

\stopsetups

\def\signaturewidth{13pc}

\startsetups tugboat:affiliation:article

  \blank[line]

  \snaptogrid \vbox \bgroup

     \forgetall

     \leftskip=\textwidth \advance\leftskip by -\signaturewidth

     \let\\=\par
     \leavevmode\llap{$\diamond$\enspace}\getvariable{tugboat}{author}
     \par
     \getvariable{tugboat}{address}
     \par
     {\tt\getvariable{tugboat}{email}}

  \egroup

\stopsetups

\startsetups tugboat:affiliation:proceedings

    % nothing fancy at the end

\stopsetups

\startsetups tugboat:article:stop

  \setups{tugboat:affiliation:\getvariable{tugboat}{type}}

  \doif{\getvariable{tugboat}{columns}}{yes}{\stopcolumns}

  \page

  \stoptext

\stopsetups

%D Normal word spacing, please.

\setuptolerance
  [strict]

%D One can use the following setups (in the article) to
%D collect settings specific to normal and/or multi
%D column typesetting.

\startsetups tugboat:columns:yes

\stopsetups

\startsetups tugboat:columns:no

\stopsetups

% >>> Logos, abbreviations: TODO: Clean up <<<
\font\mflogo = logo10
\def\MF{{\mflogo META\-FONT}}

\def\ALEPH{Aleph}
\def\API{\acro{API}}
\def\CCODE{C}
\def\CD{\acro{CD}}
\def\CMYK{\acro{CMYK}}
\def\CONTEXT{C\kern-.0333emon\-\kern-.0667em\TeX\kern-.0333emt}
\def\CPU{\acro{CPU}}
\def\DVI{\acro{DVI}}
\def\DVIPDFMX{dvipdfmx}
\def\DVIPS{dvips}
\def\ETEX{$\varepsilon$-\kern-.125em\TeX}
\def\FTP{\acro{FTP}}
\def\HTTP{\acro{HTTP}}
\def\IO{\acro{I/O}}
\def\ISO{\acro{ISO}}
\def\KPSEWHICH{kpsewhich}
\def\KPSE{\acro{KPSE}}
\newcount\TestCount
\newbox\TestBox
\def\La{\TestCount=\the\fam \leavevmode L%
        \setbox\TestBox=\hbox{$\fam\TestCount\scriptstyle A$}%
        \kern-.5\wd\TestBox\raise.42ex\box\TestBox}
\def\LATEX{\La\kern-.15em\TeX}
\def\LATEXE{\LaTeX{}\kern.05em2$_{\textstyle\varepsilon}$}
\def\LINUX{Linux}
\def\LPEG{Lpeg}
\def\LUAJIT{Lua\acro{JIT}}
\def\LUATEX{Lua\-\TeX}
\def\LUATOOLS{lua\-tools}
\def\LUA{Lua}
\def\MATHML{Math\acro{ML}}
\def\METAFUN{Metafun}
\def\METAPOST{MetaPost}
\def\METATEX{Meta\TeX{}}
\def\MKII{Mk\acro{II}}
\def\MKIV{Mk\acro{IV}}
\def\MPLIB{\acro{MP}lib}
\def\MPTOPDF{mp\-to\-pdf}
\def\MSWINDOWS{Windows}
\def\MTXTOOLS{mtx\-tools}
\def\NFSS{\acro{NFSS}}
\def\OPENMATH{Open\-Math}
\def\OPENTYPE{Open\-Type}
\def\PASCAL{Pascal}
\def\PDFTEX{pdf\/\-\TeX}
\def\PDF{\acro{PDF}}
\def\POSIX{\acro{POSIX}}
\def\PRAGMA{Pragma \acro{ADE}}
\def\POSTSCRIPT{Post\-Script}
\def\RGB{\acro{RGB}}
\def\RUBY{Ruby}
\def\SCITE{Sci\acro{TE}}
\def\TDS{\acro{TDS}}
\def\TEXBOOK{{\sl The \TeX book}}
\def\TEXEXEC{\TeX exec}
\def\TEX{\TeX}
\def\TFM{\acro{TFM}}
\def\TRUETYPE{True\-Type}
\def\TYPEONE{Type~1}
\def\UTF{\acro{UTF}}
\def\WEBC{Web2C}
\def\XETEX{X\lower.5ex\hbox{\kern-.15em\mirror{E}}\kern-.1667em\TeX}
\def\XML{\acro{XML}}
\def\XPATH{\acro{XP}ath}
\def\XSLT{\acro{XSLT}}
\def\XSLTPROC{\acro{XSLTPROC}}
\def\ZIP{zip}

\def\Dash{\unskip\thinspace---\thinspace\ignorespaces}
\def\slash{/\penalty0 \hskip0pt \relax}

\definefont[AcroFont][Serif sa .91]
\def\acro#1{{\AcroFont #1}}

\lefthyphenmin=2 \righthyphenmin=3 % disallow x- or -xx breaks

\hyphenation{Post-Script data-base data-bases}

% hack to read tugboat.dates settings.
\def\vol#1, #2.{\def\tubvol{#1}\def\tubnum{#2}}
\def\issyear#1.{\def\tubyear{#1}}
\newcount\issueseqno

% >>>
%D Good bye. <<<

\doifnotmode{demo}{\endinput}

\showgrid

\starttext

\StartArticle

\StartAbstract
    \input bryson
\StopAbstract

\dorecurse{30}{\input ward \endgraf} \page

\startitemize
\item \input ward
\item \input ward
\stopitemize

\StopArticle

\setvariables[tugboat][columns=yes]

\StartArticle

\StartAbstract
    \input bryson
\StopAbstract

\dorecurse{30}{\input ward \endgraf} \page

\StopArticle

\stoptext

% >>>

% vim: foldmethod=marker foldmarker=<<<,>>>
