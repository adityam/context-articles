\usemodule[abr-03]
\usemodule[tugboat]

% \CONTEXT\ uses some weird rules for breaking urls. A small attempt to sanitize
% that

\sethyphenatedurlafter /
\sethyphenatedurlafter .

\setvariables
  [tugboat]
  [type=article,
   columns=yes,
   grid=no]

%\setvariables
%   [tugboat]
%   [year=\tubyear,
%    volume=\tubvol,
%    number=\tubnum,
%    page=\tubpage]

\setvariables
  [tugboat]
  [   title={\CONTEXT\ basics for users: Using images},
     author={Aditya Mahajan},
    address={},
      email={adityam (at) ieee (dot) org},
  ]

% AM: It is distracting to add width in all examples. So, I prespecify the width
% of the images. I use the figure name as the figure label; so that the reader
% of the document would not notice that we are using figure labels.
\useexternalfigure[http://www.tug.org/images/tuglogo.png]
                  [http://www.tug.org/images/tuglogo.png]
                  [scale=2000]

\starttext
\StartAbstract
  As the cliché goes, a picture is worth a thousand words. This article provides
  an overview of inserting pictures or images in a \CONTEXT\ document. 
\StopAbstract

\StartArticle

\subject {A note about MkII and MkIV}

In contrast to the previous article in this series, from now on, I assume that a
\CONTEXT\ user is using \CONTEXT\ \MKIV: \LUATEX\ engine and \PDF\ output.
\CONTEXT\ \MKIV\ behaves differently from \MKII, and in most cases
provides additional features that are absent from \MKII. 

\section {Basic usage}

The simplest way to insert an image is to use:
\startbuffer
\externalfigure[context-logo.pdf]
\stopbuffer

\typebuffer

\midaligned{\getbuffer}

This command places the \PDF\ image \filename{context-logo.pdf} in a
\type{\vbox}; the width and height of the image are equal to the natural
dimensions of the image.

To set the width of the image to a specific size, say \type{1cm}, use:
\starttyping
\externalfigure[context-logo.pdf][width=1cm]
\stoptyping

\noindentation
Similarly, to set the height of the image to a specific size, say \type{2cm}, use:
\starttyping
\externalfigure[context-logo.pdf][height=2cm]
\stoptyping

If only the \type{width} or \type{height} of the image is specified, the other
dimension is scaled appropriately to keep the aspect ratio.

To include a specific page, say page 5, of a multi-page \PDF\ file, use:
\starttyping
\externalfigure[context-logo.pdf][page=5]
\stoptyping

These four variations cover 90\% of the use cases.

\subsection[sec:native] {Natively supported file formats}

\CONTEXT\ natively supports the image formats enumerated below. The image format
is determined from the file extension (in a case insensitive manner).
% AM: Should we change this to a table?
\startitemize[nowhite]
  \item \cap{PDF}: File extension \filename{.pdf}
  \item \cap{MPS} (Metapost output): File extension \filename{.mps} or
    \filename{.[digit]}.
  \item \cap{JPEG}: File extension \filename{.jpg} or \filename{.jpeg}.
  \item \cap{PNG}: File exetension \filename{.png}
  \item \cap{JPEG 2000}: File extension \filename{.jp2}
  \item \cap{JBIG} or \cap{JBIG2}: File extension \filename{.jbig},
    \filename{.jbig2}, or \filename{.jb2}.
\stopitemize

\subsection {Including images after conversion}

The image formats listed in \in{Sec.}[sec:native] are the ones that may be
embedded directly in a \PDF. \CONTEXT\ also supports a few other formats which
are first converted to \PDF\ using an external program. Of course, for such an
conversion to work, the corresponding converter must be in the \type{PATH}.

% Of course, this
% conversion works only if the corresponding external program is listed in
% \type{content.directives.system.commandlist} list in \type{texmfcnf.lua}.
% % AM: Is texmfcnf.lua documented anywhere?
% If you use the \type{texmfcnf.lua} file that comes with \TEXLIVE\ or \CONTEXT\
% standalone, you don't need to worry about this.

\startsetups table
  \setupTABLE[frame=off, align=middle]
  \setupTABLE[row][first][topframe=on, framethickness=1.2pt, style=bold]
  \setupTABLE[row][2][topframe=on, framethickness=0.8pt, loffset=0.5em, roffset=0.5em]
  \setupTABLE[row][last][bottomframe=on, framethickness=1.2pt]
  \setupTABLE[column][1][style=\cap]
  \setupTABLE[column][3][align=flushleft, ]
\stopsetups

%AM: I don't know what the inkscape cell is not formatted correctly.
\blank[halfline]
\startTABLE[setups=table]
  \NC Format \NC Extension     \NC Converter \NC \NR
  \NC SVG    \NC \mono{.svg}, \mono{.svgz}  \NC \null\hskip-0.7em\mono{inkscape} \NC \NR
  \NC EPS    \NC \mono{.eps}, \mono{.ai}    \NC \mono{gs} (or  \mono{gswin32c}
      on Windows) from Ghostscript. \NC \NR
  \NC GIF    \NC \mono{.gif}  \NC \mono{gm} from GraphicMagick  \NC \NR
  \NC TIFF    \NC \mono{.tiff}  \NC \mono{gm} from GraphicMagick  \NC \NR
\stopTABLE
\blank[halfline]

The conversion generates a \PDF\ file with prefix \type{m_k_i_v_} and a suffix
\type{.pdf} added to the name of the original file. The result is cached, and
the conversion is rerun if the timestamps of the original file is newer than that
of the converted file.



\subsection {Specifying image directories}


By default, \CONTEXT\ searches a file in the current directory, the parent
directory, the grand-parent directory, and the \type{TEXMF} tree. 

To search for the files in other directories, say \type{images/} subdirectory
and \type{/home/user/images} directory, use:
\startbuffer
\setupexternalfigures
      [directory={images, /home/user/images}]
\stopbuffer

\typebuffer

Note that one should always use forward slashes in path names, {\em even on
Windows}.

% AM: Find out!
% To disable searching of files in the texmf tree, use

\subsection {Include remote images}

Like all other \CONTEXT\ macros that read files, \type{\externalfiguure} also
supports reading remote files from HTTP/HTTPS servers. 

\startbuffer
\externalfigure[http://www.tug.org/images/tuglogo.png]
\stopbuffer
\typebuffer

\midaligned{\getbuffer} 

When a document containing a remote file is compiled for the first time, the
remote file is downloaded from the server and stored in the \LUATEX\ cache
directory. This cached file is used during subsequent runs.

% AM: Should we go into details about how to force redownloading of files?

\section {Transform images}

\subsection {Scale images}

To scale an image use the \type{scale} key: \type{scale=1000} corresponds to the
original dimensions of the image; \type{scale=500} scales the image to
\type{50%} of the original size; \type{scale=1500} scales the images to
\type{150%} of its original size; and so on. For example: 

\startbuffer
\externalfigure[context-logo.pdf][scale=500]
\stopbuffer
\typebuffer

\midaligned{\getbuffer}

Use \type{\setupexternalfigures} to set the scale of all images. For example,
to scale all images to be twice their original size, use:
\starttyping
\setupexternalfigures
    [scale=2000]
\stoptyping

In addition, the \type{xscale} and \type{yscale} keys scale the image only in
one dimension. For example:

\startbuffer[x]
\externalfigure[context-logo.pdf][xscale=500]
\stopbuffer
%\typebuffer

\startbuffer[y]
\externalfigure[context-logo.pdf][yscale=500]
\stopbuffer
%\typebuffer

\midaligned\bgroup
  \setupcombination[distance=3em]
  \startcombination[2]
    {\getbuffer[x]}{\type{xscale=500}}
    {\getbuffer[y]}{\type{yscale=500}}
  \stopcombination
\egroup

If either \type{width} or \type{height} is specified, then the \type{scale} key
has no affect. 

\subsection {Specify maximum size of an image}

Oftentimes, we don't want the included image to be greater than a particular
size. For example, to ensure that an included image is not more than
\type{0.25\textwidth}, use:

\startbuffer
\externalfigure[cow.pdf][maxwidth=0.25\textwidth]
\stopbuffer
\typebuffer

\midaligned{\getbuffer}

If \type{maxwidth} is specified and the width of the image is less than
\type{maxwidth}, then the image is not scaled; if the width of the image is
greater than \type{maxwidth}, then the width is restricted to \type{maxwidth}
and the height is scaled appropriately to maintain the original aspect ratio.

The option \type{maxheight} is similar to \type{maxwidth} but checks the height
of the image rather than the width. 
% For example:
% \startbuffer
% \externalfigure[cow.pdf][maxheight=2cm]
% \stopbuffer
% \typebuffer
% 
% \midaligned{\getbuffer}

In most of my style files, I usually set:
\starttyping
\setupexternalfigures
    [ maxwidth=\textwidth,
      maxheight=0.8\textheight]
\stoptyping
\noindentation
to ensure that the figures do not overflow the text area.

\subsection {Rotate images}

To rotate the included image by 90, 180, or 270 degrees, use the
\type{orientation} key. For example:
\startbuffer
\externalfigure[context-logo.pdf][orientation=90]
\stopbuffer
\typebuffer

\midaligned{\getbuffer}

To rotate by an arbitrary angle, use the \type{\rotate} command. For example:
% AM: Interesting anti-aliasing at 100% zoom on evince.
% Goes away at 300% zoom.
\startbuffer
\rotate[rotation=45]{\externalfigure[context-logo.pdf]}
\stopbuffer
\typebuffer

% AM: Rotated image takes additional space because the bounding box of the image
% is rotated. So, I manually remove the extra space
\vskip -\baselineskip
\midaligned{\getbuffer}
\vskip -\baselineskip

% TODO: Update the wiki page on rotate and add a reference to it.

\subsection {Mirror images}

The \type{\externalfigure} macro does not provide any support for flipping
images. Use the generic \type{\mirror} command to mirror images. For example, to
mirror horizontally use:
\startbuffer
\mirror{\externalfigure[context-logo.pdf]}
\stopbuffer
\typebuffer

\midaligned{\getbuffer}

\noindentation and to mirror vertically, first rotate the image by 180° and then
mirror it:
\startbuffer
\mirror{\externalfigure[context-logo.pdf]
                       [orientation=180]}
\stopbuffer
\typebuffer

\midaligned{\getbuffer}

% TODO: Add reference to the wiki page on mirror

\subsection {Clip images}

The \type{\externalfigure} macro does not provide any support for clipping
images. Use the generic \type{\clip} command to clip images. 

For example, to clip the original image to a \type{1cm x 2cm}
rectangle at an offset of \type{(3mm, 5mm)} from the top left corner, use:
\startbuffer
\clip[width=1cm, height=2cm, hoffset=3mm, voffset=5mm]
     {\externalfigure[context-logo.pdf]}
\stopbuffer
\typebuffer

\midaligned{\getbuffer}

As another example, to cut the image into a \type{3x3} pieces and include the
\type{(2,2)} piece, use:
\startbuffer
\clip[nx=3,ny=3,x=2,y=2]
     {\externalfigure[context-logo.pdf]}
\stopbuffer
\typebuffer

\midaligned{\getbuffer}

% TODO: Add details to the wiki page on clip and refer to that.

\section {Troubleshooting}

\subsection {Visualize the bounding box of the image}

In some situation, e.g., if the image is taking more space than expected, it is
useful to visualize the bounding box of the image. To do so, use:
\startbuffer
\externalfigure[context-logo.pdf][frame=on]
\stopbuffer
\typebuffer

\midaligned{\getbuffer}

\CONTEXT\ includes a Perl script \type{pdftrimwhite} that may be used to
remove extra white space around a \PDF\ file. To run this script, type:
\startlines\switchtobodyfont[small]\tt
mtxrun --script pdftrimwhite {\it [flags]} {\it input} {\it output}
\stoplines

\noindentation

The most important flag is \type{--offset=<dimen>} that keeps some extra
space around the trimmed image.

A similar functionality is provided by another Perl script \type{pdfcrop} that
is part of most \TEX\ distributions.


\subsection {Tracking what is happening}

To get diagnostic information about inclusion of images, enable the tracker
\type{graphics.locating} by either adding
\starttyping
\enabletrackers[graphics.locating]
\stoptyping
\noindentation
in the \CONTEXT\ file, or by compiling the \CONTEXT\ file using the command
\startlines\switchtobodyfont[small]\tt
context {\sl --trackers=graphics.locating} filename
\stoplines

\indentation
The \type{graphics.locating} tracker outputs diagnostic information to the
console. Suppose we use \type{\externalfigure[somefile.pdf]} and \CONTEXT\ finds
the file in the current search path, then the following information is printed
on the console:
% AM: Urgh. Footnotes don't work properly in TL'12
(The tracking messages are formatted to typeset
nicely. The actual messages are displayed differently.)
\starttyping
graphics > inclusion > locations: local,global
graphics > inclusion > path list: . .. ../..
graphics > inclusion > strategy: forced format pdf
graphics > inclusion > found: somefile.pdf -> 
                                      somefile.pdf
graphics > inclusion > format natively supported
                                    by backend: pdf
\stoptyping

If the file \type{somefile.pdf} is not found in the current
search path, then the following information is printed on the
console
%\note[formatting]
(even if \type{graphics.locating} tracker is not set)
\starttyping
graphics > inclusion > strategy: forced format pdf
graphics > inclusion > not found: somefile.pdf
graphics > inclusion > not found: ./somefile.pdf
graphics > inclusion > not found: ../somefile.pdf
graphics > inclusion > not found: ../../somefile.pdf
graphics > inclusion > not found: images/somefile.pdf
graphics > inclusion > not found: 
                       /home/user/images/somefile.pdf
graphics > inclusion > format not supported: 
\stoptyping
\noindentation
and a placeholder gray box is shown in the \PDF\ output: 

\midaligned{\externalfigure[somefile.pdf]}

\subsection {Images at the beginning of a paragraph}

An \type{\externalfigure[...]} at the beginning of a paragraph leads to a line
break after the image. This is because \type{\externalfigure} is a \type{\vbox}
and when \TEX\ encounters a \type{\vbox} at the beginning of a paragraph, it
switches to vertical-mode. To prevent this, simply add \type{\dontleavehmode}
before \type{\externalfigure}, thus:
\starttyping
\dontleavehmode\externalfigure[...] ... the first line 
\stoptyping

% \section    {Advanced features}
% \subsection {Omit the file extension}
% \subsection {useexternalfigure}
% \subsection {defineexternalfigure}

\section {Conclusion}

In this article, I briefly explained how to include images in your document. In
most cases, one wants to the included images to have a number and a
caption---i.e., to display the image in a float. I will discuss floats in one of
the future articles in this series.


\StopArticle
\stoptext
