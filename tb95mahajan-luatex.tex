% engine=xetex
%AM: I really dislike the "smart" quotes in TeX. XeTeX and LuaTeX 
%do not change ' to ‘. Ideally, I would have liked to process this 
%article with LuaTeX but single-column footnotes do not work with MkIV.
%As a compromise, I am using XeTeX. If anything fails, just remove the
%above line. I can live with the smart quotes.

\usemodule[abr-03]
\usemodule[tugboat]

\input tugboat.dates
\edef\tubpage{\the\count0}

%\hyphenation{stop-lua-code}

\setupcolors[state=start]

\setvariables
  [tugboat]
  [type=article,
   columns=yes,
   grid=no]

 \setvariables
    [tugboat]
    [year=\tubyear,
     volume=\tubvol,
     number=\tubnum,
     page=\tubpage]

\setvariables
  [tugboat]
  [   title={\LUATEX: A user's perspective},
     author={Aditya Mahajan},
    address={},
      email={adityam (at) umich dot edu},
  ]


% \setuppapersize[letter][letter]


\StartAbstract
In this article, I explain how to use \LUA\ to write macros in \LUATEX. I give
some examples of macros that are complicated in \PDFTEX, but can be defined
easily using \LUA\ in \LUATEX. These examples include macros that do arithmetic
on their arguments, use loops, and parse their arguments.
\StopAbstract

\starttext
\StartArticle 

\section {Introduction}

\TEX\ is getting a new engine\Dash\LUATEX. As its name suggests, \LUATEX\ adds
\LUA, a programming language, to \TEX, the typesetter. I cannot overemphasize
the significance of being able to program \TEX\ in a high-level programming
language. For example, consider a \TEX\ macro that divides two numbers. Such a
macro is provided by the \filename{fp} package and also by \filename{pgfmath}
library of the \filename{TikZ} package. The following comment is from the
\filename{fp} package
\starttyping
\def\FP@div#1#2.#3.#4\relax#5.#6.#7\relax{%
 % [...] algorithmic idea (for x>0, y>0)
 %   - %determine \FP@shift  such that 
 %      y*10^\FP@shift < 100000000
 %                     <=y*10^(\FP@shift+1)
 %   - %determine \FP@shift' such that 
 %      x*10^\FP@shift'< 100000000
 %                     <=x*10^(\FP@shift+1)
 %   - x=x*\FP@shift'
 %   - y=y*\FP@shift
 %   - \FP@shift=\FP@shift-\FP@shift'
 %   - res=0
 %   - while y>0 %fixed-point representation!
 %   -   \FP@times=0
 %   -	  while x>y
 %   -     \FP@times=\FP@times+1
 %   -     x=x-y
 %   -   end
 %   -   y=y/10
 %   -   res=10*res+\FP@times/1000000000
 %   - end
 %   - %shift the result according to \FP@shift
\stoptyping

\noindentation
The \filename{pgfmath} library implements the macro in a similar way, but limits
the number of shifts that it does. These macros highlight the state of affairs in
writing \TEX\ macros. Even simple things like multiplying two numbers are hard;
you either have to work extremely hard to circumvent the programming
limitations of \TEX, or, more frequently, hope that someone else has done the
hard work for you. In \LUATEX, such a function can be written using the
\type{/} operator (I will explain
the details later):
\starttyping
\def\DIVIDE#1#2{\directlua{tex.print(#1/#2)}}
\stoptyping

Thus, with \LUATEX\ ordinary users can write simple macros; and, perhaps more
importantly, can read and understand macros written by \TEX\ wizards.

Since the \LUATEX\ project started it has been actively supported by \CONTEXT.
\footnote{Not surprising, as two of \LUATEX's main developers\Dash Taco
Hoekwater and Hans Hagen\Dash are also the main \CONTEXT\ developers.} These days,
the various \quotation{How do I write such a macro} questions on the \CONTEXT\
mailing list are answered by a solution that uses \LUA. I present a few such
examples in this article. I have deliberately avoided examples about fonts and
non-Latin languages. There is already quite a bit of documentation about them.
In this article, I want to highlight how to use \LUATEX\ to write macros that
require some \quotation{flow control}: randomized outputs, loops, and parsing.


\section {Interaction between \TEX\ and \LUA}

To a first approximation, the interaction between \TEX\ and \LUA\ is
straightforward. When \TEX\ (i.e., the \LUATEX\ engine) starts, it loads the
input file in memory and processes it token by token. When \TEX\ encounters
\type{\directlua}, it stops reading the file in memory, {\em fully expands the
argument of\/ \type{\directlua}}, and passes the control to a \LUA\ instance. The
\LUA\ instance, which runs with a few preloaded libraries, processes the
expanded arguments of \type{\directlua}. This \LUA\ instance has a special
output stream which can be accessed using \type{tex.print(...)}. The function
\type{tex.print(...)} is just like the \LUA\ function \type{print(...)} except
that \type{tex.print(...)} prints to a \quotation{\TEX\ stream} rather than to
the standard output. When the \LUA\ instance finishes processing its input, it
passes the contents of the \quotation{\TEX\ stream} back to \TEX.\footnote{The
output of \type{tex.print(...)} is buffered and not passed to \TEX\ until the
\LUA\ instance has stopped.} \TEX\ then inserts the contents of the
\quotation{\TEX\ stream} at the current location of the file that it was
reading; expands the contents of the \quotation{\TEX\ stream}; and continues.
If \TEX\ encounters another \type{\directlua}, the above process is repeated. 

As an exercise, imagine what happens when the following input is processed by
\LUATEX. 
\footnote{In this example, I used two different kinds of quotations to
avoid escaping quotes. Escaping quotes inside  \type{\directlua} is tricky.
The above was a contrived example; if you ever need to escape quotes, you can
use the \type{\startluacode ... \stopluacode} syntax explained later.}

\starttyping
\directlua%
  {tex.print("Depth 1 
           \\directlua{tex.print('Depth 2')}")}
\stoptyping

On top of these \LUATEX\ primitives, \CONTEXT\ provides a higher level
interface. There are two ways to call \LUA\ from \CONTEXT. The first is a macro
\type{\ctxlua} (read as \CONTEXT\ \LUA), which is similar to \type{\directlua}.
(Aside: It is possible to run the \LUA\ instance under different name spaces.
\type{\ctxlua} is the default name space; other name spaces are explained
later.) \type{\ctxlua} is good for calling small snippets of \LUA. The argument
of \type{\ctxlua} is parsed under normal \TEX\ catcodes (category
codes), so the end
of line character has the same catcode as a space. This can lead to surprises.
For example, if you try to use a \LUA\ comment, everything after the comment
gets ignored.
\starttyping
\ctxlua
  {-- A lua comment
   tex.print("This is not printed")}
\stoptyping

This can be avoided by using a \TEX\ comment instead of a \LUA\ comment.
However, working under normal \TEX\ catcodes poses a bigger problem: special
\TEX\ characters like \letterampersand, \letterhash, \letterdollar, \{, \}, etc.,
need to be escaped. For example, \letterhash\ has to be escaped with
\type{\string} to be used in \type{\ctxlua}.
\starttyping
\ctxlua
  {local t = {1,2,3,4}
   tex.print("length " .. \string#t)}
\stoptyping

As the argument of \type{\ctxlua} is fully expanded, escaping
characters can sometimes be tricky. To circumvent this problem, \CONTEXT\
defines a  environment called \type{\startluacode ... \stopluacode}. This sets
the catcodes to what one would expect in \LUA. Basically only \type{\}
has its usual \TEX\
meaning, the catcode of everything else is set to other. So, for all practical
purposes, we can forget about catcodes inside \type{\startluacode ...
\stopluacode}. The above two examples can be written as
\starttyping
\startluacode
  -- A lua comment
   tex.print("This is printed.")
  local t = {1,2,3,4}
   tex.print("length " .. #t)
\stopluacode
\stoptyping

This environment is meant for moderately sized code snippets. For longer \LUA\
code, it is more convenient to write the code in a separate \LUA\ file and then
load it using \LUA's \type{dofile(...)} function. 

\CONTEXT\ also provides a \LUA\ function to conveniently write to the \TEX\
stream. The function is called \type{context(...)} and it is equivalent to
\type{tex.print(string.format(...))}. 

Using the above, it is easy to define \TEX\ macros that pass control to
\LUA, do some processing in \LUA, and then pass the result back to \TEX. For
example, a macro to convert a decimal number to hexadecimal can be
written simply, by asking \LUA\ to do the conversion.
\starttyping
\def\TOHEX#1{\ctxlua{context("\%X",#1)}}
\TOHEX{35}
\stoptyping

\noindentation
The percent sign had to be escaped because \type{\ctxlua} assumes
\TEX\ catcodes. Sometimes, escaping arguments can be difficult; instead,
it can be
easier to define a \LUA\ function inside \type{\startluacode ... \stopluacode}
and call it using \type{\ctxlua}. For example, a macro
that takes a comma separated list of strings and prints a random item can be
written as 
\starttyping
\startluacode
  userdata = userdata or {}
  math.randomseed( os.time() )
  function userdata.random(...)
    context(arg[math.random(1, #arg)])
  end
\stopluacode
  
\def\CHOOSERANDOM#1%
  {\ctxlua{userdata.random(#1)}}

\CHOOSERANDOM{"one", "two", "three"}
\stoptyping

I could have written a wrapper so that the function takes a list of words and
chooses a random word among them. For an example of such a conversion, see
the \quotation{sorting a list of tokens} page on the \LUATEX\
wiki~\cite[luatex-wiki].

In the above, I created a name space called \type{userdata} and defined the
function \type{random} in that name space. Using a name space avoids clashes
with the \LUA\ functions defined in \LUATEX\ and \CONTEXT.

In order to avoid name clashes, \CONTEXT\ also defines independent name spaces of
\LUA\ instances. They are 
\startnarrower
\setuptabulate[before={\blank[small]}, after={\blank[medium]}]
\starttabulate[|rT|p|]
  \NC user     \EQ a private user instance        \NC\AR
  \NC third    \EQ third party module instance    \NC\AR
  \NC module   \EQ \CONTEXT\ module instance      \NC\AR
  \NC isolated \EQ an isolated instance           \NC\AR
\stoptabulate
\stopnarrower

Thus, for example, instead of \type{\ctxlua} and \type{\startluacode ...
\stopluacode}, the \type{user} instance can be accessed via the macros
\type{\usercode}
and \type{\startusercode ... \stopusercode}. In instances other than
\type{isolated}, all the \LUA\ functions defined by \CONTEXT\ (but
not the inbuilt \LUA\ functions) are stored in a \type{global} name space. In
the \type{isolated} instance, all \LUA\ functions defined by \CONTEXT\ are
hidden and cannot be accessed. Using these instances, we could
write the above \type{\CHOOSERANDOM} macro as follows
\starttyping
\startusercode
  math.randomseed( global.os.time() )
  function random(...)
    global.context(arg[math.random(1, #arg)])
  end
\stopusercode

\def\CHOOSERANDOM#1%
  {\usercode{random(#1)}}
\stoptyping

Since I defined the function \type{random} in the \type{user}
instance of \LUA, I did not bother to use a separate name space for the
function. The \LUA\ functions \type{os.time}, which is defined by a \LUATEX\
library, and \type{context}, which is defined by \CONTEXT, needed to be accessed
through a \type{global} name space. On the other hand, the
\type{math.randomseed} function, which is part of \LUA, could be accessed as is. 

A separate \LUA\ instance also makes debugging slightly easier. With
\type{\ctxlua} the error message starts with
\starttyping
! LuaTeX error <main ctx instance>:
\stoptyping

\noindentation
With \type{\usercode} the error message starts with
\starttyping
! LuaTeX error <private user instance>:
\stoptyping

\noindentation This makes it easier to narrow down the source of error. 

Normally, it is best to define your \LUA\ functions in the \type{user} name
space. If you are writing a module, then define your \LUA\ functions in the
\type{third} instance and in a name space which is the name of your module. In
this article, I will simply use the default \LUA\ instance, but take care to
define all my \LUA\ functions in a \type{userdata} name space.

Now that we have some idea of how to work with \LUATEX, let's look at some
examples.

\section {Arithmetic without using a abacus}

Doing simple arithmetic in \TEX\ can be extremely difficult, as illustrated by
the division macro in the introduction. With \LUA, simple arithmetic becomes
trivial. For example, if you want a macro to find the cosine of an angle (in
degrees), you can write
\starttyping
\def\COSINE#1%
  {\ctxlua(context(math.cos(#1*2*pi/360))}
\stoptyping

The built-in \type{math.cos} function assumes that the argument is specified in
radians, so we convert from degrees to radians on the fly. If you want to
type the value of $\pi$ in an article, you can simply say
\starttyping
$\pi = \ctxlua{context(math.pi)}$ 
\stoptyping
\noindentation or if you want less precision (notice the percent sign is
escaped)
\starttyping 
$\pi = \ctxlua{context("\%.6f", math.pi)}$
\stoptyping


\section {Loops without worrying about expansion}

\startbuffer[table-setup]
\setupTABLE[each][each][width=2em,height=2em,align={middle,middle}]  
\setupTABLE[r][1][background=color,backgroundcolor=gray]  
\setupTABLE[c][1][background=color,backgroundcolor=gray]
\stopbuffer

\getbuffer[table-setup]

\startbuffer[table-type]
\bTABLE  
  \bTR \bTD $(+)$ \eTD \bTD 1 \eTD \bTD 2 \eTD 
       \bTD 3 \eTD \bTD 4  \eTD \bTD 5  \eTD \bTD 6  \eTD \eTR  
  \bTR \bTD 1     \eTD \bTD 2 \eTD \bTD 3 \eTD 
       \bTD 4 \eTD \bTD 5  \eTD \bTD 6  \eTD \bTD 7  \eTD \eTR  
  \bTR \bTD 2     \eTD \bTD 3 \eTD \bTD 4 \eTD 
       \bTD 5 \eTD \bTD 6  \eTD \bTD 7  \eTD \bTD 8  \eTD \eTR  
  \bTR \bTD 3     \eTD \bTD 4 \eTD \bTD 5 \eTD
       \bTD 6 \eTD \bTD 7  \eTD \bTD 8  \eTD \bTD 9  \eTD \eTR  
  \bTR \bTD 4     \eTD \bTD 5 \eTD \bTD 6 \eTD
       \bTD 7 \eTD \bTD 8  \eTD \bTD 9  \eTD \bTD 10 \eTD \eTR  
  \bTR \bTD 5     \eTD \bTD 6 \eTD \bTD 7 \eTD 
       \bTD 8 \eTD \bTD 9  \eTD \bTD 10 \eTD \bTD 11 \eTD \eTR  
  \bTR \bTD 6     \eTD \bTD 7 \eTD \bTD 8 \eTD 
       \bTD 9 \eTD \bTD 10 \eTD \bTD 11 \eTD \bTD 12 \eTD \eTR  
\eTABLE
\stopbuffer

Loops in \TEX\ are tricky because macro assignments and macro expansion interact
in strange ways. For example, suppose we want to typeset a table showing the sum
of the roll of two dice and want the output to look like this
\placetable
    [here,none]
    {none}
    {\getbuffer[table-type]}

The tedious (but  faster!)\ way to achieve this is to simply type the whole table
by hand. For example, 
\starttyping
\bTABLE
  \bTR \bTD $(+)$ \eTD \bTD 1 \eTD ... ... \eTR
  \bTR \bTD   1   \eTD \bTD 2 \eTD ... ... \eTR
  ...   ...   ...   ...   ...   ...
  ...   ...   ...   ...   ...   ...
\eTABLE
\stoptyping

It is however natural to want to write this table as a loop, and compute
the values. A first \CONTEXT\
implementation using the recursion level might be:
\startbuffer[expansion-1]
\bTABLE  
 \bTR  
  \bTD $(+)$ \eTD  
  \dorecurse{6}  
   {\bTD \recurselevel \eTD}  
  \eTR  
\dorecurse{6}  
 {\bTR  
     \bTD \recurselevel \eTD  
     \edef\firstrecurselevel{\recurselevel}  
  \dorecurse{6}  
  {\bTD 
   \the\numexpr\firstrecurselevel+\recurselevel 
  \eTD}%
  \eTR}  
\eTABLE
\stopbuffer

\typebuffer[expansion-1]

\noindentation However, this does not work as expected, yielding all zeros.
%
%% Darn...pdftex and xetex behave differently here. So, cheating.
%\def\recurselevel{0}
%\placetable
%    [here,none]
%    {none}
%    {\getbuffer[expansion-1]}
%
A natural table stores the contents of all the cells, before typesetting
it. But
it does not expand the contents of its cell before storing them. So, at the time
the table is actually typeset, \TEX\ has already finished the \type{\dorecurse}
and \type{\recurselevel} is set to 0.

The solution is to place
\type{\expandafter} at the correct location(s) to coax \TEX\ into expanding
the \type{\recurselevel} macro before the natural table stores the cell contents. The
difficult part is figuring out the exact location of \type{\expandafter}s. Here
is a solution that works:

\startbuffer[expansion-2]
\bTABLE  
 \bTR  
  \bTD $(+)$ \eTD  
  \dorecurse{6}  
   {\expandafter \bTD \recurselevel \eTD}  
  \eTR  
\dorecurse{6}  
{\bTR  
 \edef\firstrecurselevel{\recurselevel}  
 \expandafter\bTD \recurselevel \eTD  
 \dorecurse{6}  
 {\expandafter\bTD
  \the\numexpr\firstrecurselevel+\recurselevel
  \relax
  \eTD}  
 \eTR}  
\eTABLE  
\stopbuffer

\typebuffer[expansion-2]

% \placetable
%     [here,none]
%     {none}
%     {\getbuffer[expansion-2]}

We only needed to add three \type{\expandafter}s to make the naive loop work.
Nevertheless, finding the right location of \type{\expandafter} can be
frustrating, especially for a non-expert.

\hfuzz=1pt
By contrast, in \LUATEX\ writing loops is easy.
Once a \LUA\ instance starts, \TEX\ does not see anything until the \LUA\
instance exits. So, we can write the loop in \LUA, and simply print the values
that we would have typed to the \TEX\ stream. When the control is passed to
\TEX, \TEX\ sees the input as if we had typed it by hand. Consequently, macro
expansion is no longer an issue. For example, we can get the above table by:
\starttyping
\startluacode
context.bTABLE()
  context.bTR()
    context.bTD() context("$(+)$") context.eTD()
    for j=1,6 do
      context.bTD() context(j) context.eTD()
    end
  context.eTR()
  for i=1,6 do
    context.bTR()
    context.bTD() context(i) context.eTD()
    for j=1,6 do
      context.bTD() context(i+j) context.eTD()
    end
    context.eTR()
  end
context.eTABLE()
\stopluacode
\stoptyping
%cld=http://www.pragma-ade.com/general/manuals/cld-mkiv.pdf 

The \LUA\ functions such as \type{context.bTABLE()}
and \type{context.bTR()} are  just
abbreviations for running \type{context ("\\bTABLE")},
\type{context("\\bTR")}, etc. See the
\CONTEXT\ \LUA\ document manual for more details about such
functions~\cite[cld]. The rest of the code is a simple nested for-loop that
computes the sum of two dice. We do not need to worry about macro expansion at
all!

\section {Parsing input without exploding your~head}

In order to get around the weird rules of macro expansion, writing a parser in
\TEX\ involves a lot of macro jugglery and catcode trickery. It is a black art,
one of the biggest mysteries of \TEX\ for ordinary users.

As an
example, let's consider typesetting chemical molecules in \TEX. Normally,
molecules should be typeset in text mode rather than math mode. For example,
H\low{2}SO\lohi{4}{--}, can be input as \type{H\low{2}SO\lohi{4}{--}}. Typing so
much markup can be cumbersome. Ideally, we want a macro such that we
type \type{\molecule{H_2SO_4^-}} and the macro translates this into
\type{H\low{2}SO\lohi{4}{--}}. Such a macro can be written in \TEX\ as follows.
\starttyping
\newbox\chemlowbox 
\def\chemlow#1% 
  {\setbox\chemlowbox
           \hbox{{\switchtobodyfont[small]#1}}} 

\def\chemhigh#1% 
  {\ifvoid\chemlowbox 
      \high{{\switchtobodyfont[small]#1}}% 
   \else 
      \lohi{\box\chemlowbox}
           {{\switchtobodyfont[small]#1}}
   \fi} 

\def\finishchem%
  {\ifvoid\chemlowbox\else 
      \low{\box\chemlowbox}
   \fi} 

\unexpanded\def\molecule% 
  {\bgroup 
   \catcode`\_=\active \uccode`\~=`\_ 
       \uppercase{\let~\chemlow}% 
   \catcode`\^=\active \uccode`\~=`\^ 
       \uppercase{\let~\chemhigh}% 
   \dostepwiserecurse {65}{90}{1} 
   {\catcode \recurselevel = \active 
    \uccode`\~=\recurselevel 
        \uppercase{\edef~{\noexpand\finishchem 
    \rawcharacter{\recurselevel}}}}% 
    \catcode`\-=\active \uccode`\~=`\- 
        \uppercase{\def~{--}}% 
    \domolecule }% 

\def\domolecule#1{#1\finishchem\egroup}
\stoptyping

This monstrosity is a typical \TEX\ parser. Appropriate characters need to be
made active; occasionally, \type{\lccode} and \type{\uccode} need to be set;
signaling tricks are needed (for instance, checking if \type{\chemlowbox} is
void); and then magic happens (or so it seems to a flabbergasted user). More
sophisticated parsers involve creating finite state automata, which look
even more monstrous.

With \LUATEX, things are different.  \LUATEX\ includes a general parser based on
\acro{PEG} (parsing expression grammar) called
\type{lpeg}~\cite[lpeg]. This makes writing parsers in \TEX\ much
more comprehensible. For example, the above \type{\molecule} macro can be
written as

\starttyping
\startluacode
userdata = userdata or {}

local lowercase   = lpeg.R("az")
local uppercase   = lpeg.R("AZ")
local backslash   = lpeg.P("\\")
local csname      = backslash * lpeg.P(1) 
                  * (1-backslash)^0
local plus        = lpeg.P("+") / "\\textplus "
local minus       = lpeg.P("-") / "\\textminus "
local digit       = lpeg.R("09")
local sign        = plus + minus
local cardinal    = digit^1
local integer     = sign^0 * cardinal
local leftbrace   = lpeg.P("{")
local rightbrace  = lpeg.P("}")
local nobrace     = 1 - (leftbrace + rightbrace)
local nested      = lpeg.P {leftbrace 
                  * (csname + sign + nobrace 
                  + lpeg.V(1))^0   * rightbrace}
local any         = lpeg.P(1)

local subscript   = lpeg.P("_")
local superscript = lpeg.P("^")
local somescript  = subscript + superscript

local content     = lpeg.Cs(csname + nested 
                           + sign + any)

local lowhigh    = lpeg.Cc("\\lohi{%s}{%s}") 
                   * subscript   * content 
                   * superscript * content 
                   / string.format
local highlow    = lpeg.Cc("\\hilo{%s}{%s}") 
                   * superscript * content 
                   * subscript   * content 
                   / string.format
local low        = lpeg.Cc("\\low{%s}") 
                   * subscript   * content       
                   / string.format
local high       = lpeg.Cc("\\high{%s}")
                   * superscript * content
                   / string.format
local justtext   = (1 - somescript)^1
local parser     = lpeg.Cs((csname + lowhigh 
                        + highlow + low 
                        + high + sign + any)^0)

userdata.moleculeparser = parser 

function userdata.molecule(str)
    return parser:match(str)
end
\stopluacode

\def\molecule#1%
  {\ctxlua{userdata.molecule("#1")}}
\stoptyping 

This is more verbose than the \TEX\ solution, but is easier to read and write.
With a proper parser, I do not have to use tricks to check if either one or both
\type{_} and \type{^} are present. More importantly,  anyone (once they know the
\LPEG\ syntax) can read the parser and easily understand what it does. This is
in contrast to the implementation based on \TEX\  macro jugglery which require
you to implement a \TEX\ interpreter in your head to understand.

\section {Conclusion}

\LUATEX\ is removing many \TEX\ barriers: using system fonts,
reading and writing Unicode files, typesetting non-Latin languages, among
others. However, the biggest feature of \LUATEX\ is the ability to use a
high-level programming language to program \TEX. This can potentially lower the
learning curve for programming \TEX.

In this article, I have mentioned only one aspect of programming \TEX: macros
that manipulate their input and output some text to the main \TEX\
stream. Many other
kinds of manipulations are possible: \LUATEX\ provides access to \TEX\ boxes,
token lists, dimensions, glues, catcodes, direction parameters, math parameters,
etc. The details can be found in the \LUATEX\  manual~\cite[manual].

\section {References}

{
\vskip-10pt
\spaceskip = \fontdimen2\font
\rightskip = 0pt plus4em \hyphenpenalty=10000
\hfuzz=5pt

\startbibliography[inbetween={\blank[4pt]}]
  \nonknuthmode %Makes the catcode of _ and ^ as letter
  \item[manual] \LUATEX\ reference manual, 
    \mono{http://www.luatex.org/documentation.html}
  \item[luatex-wiki] Sorting a list of tokens, in the Joy of \LUATEX. 
    \mono{http://luatex.bluwiki.com/go/\crlf Sort_a_token_list}
  \item[cld] Hans Hagen, \quotation{CLD: \CONTEXT\ \LUA\ document}, \crlf
    \mono{http://www.pragma-ade.com/general/\crlf manuals/cld-mkiv.pdf}
  \item[lpeg] \LPEG: Parsing Expression Grammars for \LUA, 
    \mono{http://www.inf.puc-rio.br/\~{}roberto/lpeg/\crlf lpeg.html}

\stopbibliography

}

% context uses count1 instead of count0 for the page number
{\count0=\count1 \writelastpage{+1}}

\StopArticle
\stoptext
